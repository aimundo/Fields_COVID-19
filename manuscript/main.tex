% Options for packages loaded elsewhere
\PassOptionsToPackage{unicode}{hyperref}
\PassOptionsToPackage{hyphens}{url}
\PassOptionsToPackage{dvipsnames,svgnames,x11names}{xcolor}
%
\documentclass[
  letterpaper,
  DIV=11,
  numbers=noendperiod]{scrartcl}

\usepackage{amsmath,amssymb}
\usepackage{lmodern}
\usepackage{setspace}
\usepackage{iftex}
\ifPDFTeX
  \usepackage[T1]{fontenc}
  \usepackage[utf8]{inputenc}
  \usepackage{textcomp} % provide euro and other symbols
\else % if luatex or xetex
  \usepackage{unicode-math}
  \defaultfontfeatures{Scale=MatchLowercase}
  \defaultfontfeatures[\rmfamily]{Ligatures=TeX,Scale=1}
  \setmainfont[]{Palatino Linotype}
\fi
% Use upquote if available, for straight quotes in verbatim environments
\IfFileExists{upquote.sty}{\usepackage{upquote}}{}
\IfFileExists{microtype.sty}{% use microtype if available
  \usepackage[]{microtype}
  \UseMicrotypeSet[protrusion]{basicmath} % disable protrusion for tt fonts
}{}
\makeatletter
\@ifundefined{KOMAClassName}{% if non-KOMA class
  \IfFileExists{parskip.sty}{%
    \usepackage{parskip}
  }{% else
    \setlength{\parindent}{0pt}
    \setlength{\parskip}{6pt plus 2pt minus 1pt}}
}{% if KOMA class
  \KOMAoptions{parskip=half}}
\makeatother
\usepackage{xcolor}
\setlength{\emergencystretch}{3em} % prevent overfull lines
\setcounter{secnumdepth}{-\maxdimen} % remove section numbering
% Make \paragraph and \subparagraph free-standing
\ifx\paragraph\undefined\else
  \let\oldparagraph\paragraph
  \renewcommand{\paragraph}[1]{\oldparagraph{#1}\mbox{}}
\fi
\ifx\subparagraph\undefined\else
  \let\oldsubparagraph\subparagraph
  \renewcommand{\subparagraph}[1]{\oldsubparagraph{#1}\mbox{}}
\fi


\providecommand{\tightlist}{%
  \setlength{\itemsep}{0pt}\setlength{\parskip}{0pt}}\usepackage{longtable,booktabs,array}
\usepackage{calc} % for calculating minipage widths
% Correct order of tables after \paragraph or \subparagraph
\usepackage{etoolbox}
\makeatletter
\patchcmd\longtable{\par}{\if@noskipsec\mbox{}\fi\par}{}{}
\makeatother
% Allow footnotes in longtable head/foot
\IfFileExists{footnotehyper.sty}{\usepackage{footnotehyper}}{\usepackage{footnote}}
\makesavenoteenv{longtable}
\usepackage{graphicx}
\makeatletter
\def\maxwidth{\ifdim\Gin@nat@width>\linewidth\linewidth\else\Gin@nat@width\fi}
\def\maxheight{\ifdim\Gin@nat@height>\textheight\textheight\else\Gin@nat@height\fi}
\makeatother
% Scale images if necessary, so that they will not overflow the page
% margins by default, and it is still possible to overwrite the defaults
% using explicit options in \includegraphics[width, height, ...]{}
\setkeys{Gin}{width=\maxwidth,height=\maxheight,keepaspectratio}
% Set default figure placement to htbp
\makeatletter
\def\fps@figure{htbp}
\makeatother
\newlength{\cslhangindent}
\setlength{\cslhangindent}{1.5em}
\newlength{\csllabelwidth}
\setlength{\csllabelwidth}{3em}
\newlength{\cslentryspacingunit} % times entry-spacing
\setlength{\cslentryspacingunit}{\parskip}
\newenvironment{CSLReferences}[2] % #1 hanging-ident, #2 entry spacing
 {% don't indent paragraphs
  \setlength{\parindent}{0pt}
  % turn on hanging indent if param 1 is 1
  \ifodd #1
  \let\oldpar\par
  \def\par{\hangindent=\cslhangindent\oldpar}
  \fi
  % set entry spacing
  \setlength{\parskip}{#2\cslentryspacingunit}
 }%
 {}
\usepackage{calc}
\newcommand{\CSLBlock}[1]{#1\hfill\break}
\newcommand{\CSLLeftMargin}[1]{\parbox[t]{\csllabelwidth}{#1}}
\newcommand{\CSLRightInline}[1]{\parbox[t]{\linewidth - \csllabelwidth}{#1}\break}
\newcommand{\CSLIndent}[1]{\hspace{\cslhangindent}#1}

%line numbers
%\usepackage{mathpazo}
\usepackage{lineno}
\linenumbers

\usepackage{booktabs}
\usepackage{longtable}
\usepackage{array}
\usepackage{multirow}
\usepackage{wrapfig}
\usepackage{float}
\usepackage{colortbl}
\usepackage{pdflscape}
\usepackage{tabu}
\usepackage{threeparttable}
\usepackage{threeparttablex}
\usepackage[normalem]{ulem}
\usepackage{makecell}
\usepackage{xcolor}
\KOMAoption{captions}{tableheading}
\makeatletter
\makeatother
\makeatletter
\makeatother
\makeatletter
\@ifpackageloaded{caption}{}{\usepackage{caption}}
\AtBeginDocument{%
\ifdefined\contentsname
  \renewcommand*\contentsname{Table of contents}
\else
  \newcommand\contentsname{Table of contents}
\fi
\ifdefined\listfigurename
  \renewcommand*\listfigurename{List of Figures}
\else
  \newcommand\listfigurename{List of Figures}
\fi
\ifdefined\listtablename
  \renewcommand*\listtablename{List of Tables}
\else
  \newcommand\listtablename{List of Tables}
\fi
\ifdefined\figurename
  \renewcommand*\figurename{Figure}
\else
  \newcommand\figurename{Figure}
\fi
\ifdefined\tablename
  \renewcommand*\tablename{Table}
\else
  \newcommand\tablename{Table}
\fi
}
\@ifpackageloaded{float}{}{\usepackage{float}}
\floatstyle{ruled}
\@ifundefined{c@chapter}{\newfloat{codelisting}{h}{lop}}{\newfloat{codelisting}{h}{lop}[chapter]}
\floatname{codelisting}{Listing}
\newcommand*\listoflistings{\listof{codelisting}{List of Listings}}
\makeatother
\makeatletter
\@ifpackageloaded{caption}{}{\usepackage{caption}}
\@ifpackageloaded{subcaption}{}{\usepackage{subcaption}}
\makeatother
\makeatletter
\@ifpackageloaded{tcolorbox}{}{\usepackage[many]{tcolorbox}}
\makeatother
\makeatletter
\@ifundefined{shadecolor}{\definecolor{shadecolor}{rgb}{.97, .97, .97}}
\makeatother
\makeatletter
\makeatother
\ifLuaTeX
  \usepackage{selnolig}  % disable illegal ligatures
\fi
\IfFileExists{bookmark.sty}{\usepackage{bookmark}}{\usepackage{hyperref}}
\IfFileExists{xurl.sty}{\usepackage{xurl}}{} % add URL line breaks if available
\urlstyle{same} % disable monospaced font for URLs
\hypersetup{
  pdftitle={COVID-19 vaccination in Ontario: Exploring intra-provincial variations within Health Regions and socio-economic strata},
  pdfauthor={Ariel Mundo Ortiz1; Bouchra Nasri2,},
  colorlinks=true,
  linkcolor={blue},
  filecolor={Maroon},
  citecolor={Blue},
  urlcolor={Blue},
  pdfcreator={LaTeX via pandoc}}

\title{\textbf{COVID-19 vaccination in Ontario: Exploring
intra-provincial variations within Health Regions and socio-economic
strata}}
\author{Ariel Mundo Ortiz\textsuperscript{1} \and Bouchra
Nasri\textsuperscript{2,*}}
\date{}

\begin{document}
\maketitle
\ifdefined\Shaded\renewenvironment{Shaded}{\begin{tcolorbox}[borderline west={3pt}{0pt}{shadecolor}, boxrule=0pt, enhanced, breakable, interior hidden, sharp corners, frame hidden]}{\end{tcolorbox}}\fi

\setstretch{2}
\textsuperscript{1} Centre de Recherches Mathématiques, University of
Montreal, Montréal, Canada\\
\textsuperscript{2} Department of Social and Preventive Medicine, École
de Santé Publique, University of Montreal, Montréal, Canada

\textsuperscript{*} Correspondence:
\href{mailto:bouchra.nasri@umontreal.ca}{Bouchra Nasri
\textless{}bouchra.nasri@umontreal.ca\textgreater{}}

\hypertarget{abstract}{%
\section{Abstract}\label{abstract}}

The COVID-19 pandemic continues to be a worldwide public health concern.
Although vaccines against this disease were rapidly developed,
vaccination uptake has not ben equal across all the segments of the
population. In particular, it has been shown that there have been
differences in vaccine uptake across different segments of the
population. However, there are also differences in vaccination across
geographical areas, which might be important to consider in the
development of future public health policies against COVID-19. In this
study, we examined the relationship between vaccination status (having
received the first dose of a COVID-19 vaccine), and different
socio-economic and geographical factors. Our results show differences in
vaccination due to race/ethnicity, income, Health Regions (geographical
areas used for health service access in Ontario), and their
interactions. In particular, we show that individuals who identified as
Arab/Middle Eastern, Black, or Latin American, had significantly lower
odds of vaccination than White/Caucasian individuals (ORs=0.31, 0.32,
0.28, and \emph{p}=0.004, \emph{p}\textless0.001 and \emph{p}=0.004,
respectively), and that individuals with a household income below CAD
25,000 who identified as Arab/Middle Eastern (OR=3.05, \emph{p}=0.013),
Black (OR=3.19, \emph{p}=0.004), Latin American (OR=2.80,
\emph{p}=0.041), or that belonged to other minority groups (OR=4.59,
\emph{p}\textless0.001) had higher odds of vaccination than individuals
from the same racial/ethnic group in higher income brackets. Finally, we
also identified lower odds of vaccination within different minority
groups in West Health Region. This study shows that there is an ongoing
need to better understand and address differences in vaccination uptake
across diverse segments of the population that have been largely
impacted by the pandemic.

\hypertarget{keywords}{%
\section*{Keywords}\label{keywords}}
\addcontentsline{toc}{section}{Keywords}

Covid-19, vaccination, survey, socio-economic factors, visible
minorities.

\hypertarget{background}{%
\section{Background}\label{background}}

The COVID-19 pandemic continues around the world with more than 600
million confirmed cases as of November of
2022\textsuperscript{\protect\hyperlink{ref-WHO-Covid}{1}}. During the
first months of the pandemic in early 2020, non-pharmaceutical
interventions such as masking and social distancing were the only
methods available to manage the spread of the disease; however, the
rapid development of vaccines permitted their approval and use in some
countries towards the last month part of 2020. For example, in the US
and Canada vaccine campaigns began in mid-December of
2020\textsuperscript{\protect\hyperlink{ref-tanne2020}{2},\protect\hyperlink{ref-bogoch2022}{3}}.
Although it has been estimated that COVID-19 vaccines have prevented
around 14 millions of deaths
worldwide\textsuperscript{\protect\hyperlink{ref-watson2022}{4}}, the
rollout of COVID-19 vaccines has faced multiple challenges since its
inception.

Indeed, multiple obstacles that have complicated vaccination efforts
against COVID-19 have been identified: inequality in vaccine access,
vaccine hesitancy, and differences in vaccination rates across different
segments of the
population\textsuperscript{\protect\hyperlink{ref-gerretsen2021}{5}--\protect\hyperlink{ref-malik2020}{7}}.
In the case of Canada, lower vaccine uptake has been associated with
socio-economic factors such as younger age, educational level, presence
of children in the household, lack of a regular healthcare provider,
ethnic origin, and financial
instability\textsuperscript{\protect\hyperlink{ref-guay2022}{8}--\protect\hyperlink{ref-hussain2022}{10}}.

Additionally, vaccination is influenced by changes in geography. In this
regard, it has been shown that there have been spatial differences in
COVID-19 vaccination rates due to regional differences in attitudes
towards
vaccination\textsuperscript{\protect\hyperlink{ref-malik2020}{7}},
geographical differences in vaccine access and supply, vaccination
location availability, and lack of prioritization of vulnerable
groups\textsuperscript{\protect\hyperlink{ref-bogoch2022}{3},\protect\hyperlink{ref-nguyen2021}{11}}.

Studies that analyze geographical variations in vaccine uptake are
important as they can help inform public health decision-makers to
design policies to that consider spatial changes to address vaccination
disparities. In this regard, previous geographical (spatial) analyses of
vaccination rates have shown that variations in vaccine uptake can occur
within small governmental administrative units (e.g., counties in the
case of the
US)\textsuperscript{\protect\hyperlink{ref-mollalo2021}{12}--\protect\hyperlink{ref-bhuiyan2022}{15}},
and that geographical analyses can be predictive of booster uptake
patterns\textsuperscript{\protect\hyperlink{ref-wood2022}{16}}.

In Canada, studies that have used a spatial approach to analyze vaccine
uptake have shown disparities in vaccination rates across low and high
income neighborhoods in the city of
Toronto\textsuperscript{\protect\hyperlink{ref-choi2021}{17}}, among
adolescents from deprived neighborhoods in the city of
Montreal\textsuperscript{\protect\hyperlink{ref-mckinnon2021}{18}}, and
have also highlighted disparities in vaccination status depending on
age, income, and ethnic origin at the provincial
level\textsuperscript{\protect\hyperlink{ref-guay2022}{8}}. However,
there is limited information on differences in vaccination status inside
the provinces. Such analysis is important as it can help identify
inequalities that may exist within these geographical areas while
providing a granular view of intra-provincial differences that can help
understand the barriers for vaccine uptake.

In this study, we examined self-reported COVID-19 vaccination status
within the province of Ontario using a combination of socio-economic
factors(such as ethnic origin, age, and income) and geographical
analysis (at the level of the Health Regions) in order to determine
intra-provincial differences in vaccine uptake and address the ongoing
need of socio-economic information that can provide a rationale for the
disparities in vaccination observed within some racial
groups\textsuperscript{\protect\hyperlink{ref-cnat2022b}{19}}.

\hypertarget{methods}{%
\section{Methods}\label{methods}}

\hypertarget{data-source-survey-overview}{%
\subsection{Data source: survey
overview}\label{data-source-survey-overview}}

We used data from the \emph{Survey of COVID-19 related Behaviours and
Attitudes}, a repeated cross sectional survey focused on the Canadian
province of Ontario which ran from Sept 30, 2021 until January 17, 2022
and that was comissioned by the Fields Institute for Research in
Mathematical Sciences (henceforth Fields) and the Mathematical Modelling
of COVID-19 Task Force. The survey was conducted by a third-party
service provider (RIWI Corp.), under ethical guidance from the
University of Toronto.

Briefly, the survey was deployed used random domain intercept
technology, where if users clicked on a registered but commercially
inactive web link or typed in a web address for a site that was dormant,
they had a random chance of that link being temporarily managed by the
company that administered the survey and instead of coming across a
notification about the status of the site(``this page does not exist''),
the survey was deployed to the
user\textsuperscript{\protect\hyperlink{ref-sargent2022}{20}}. Users
then decided whether to anonymously participate, and those that
participated were able to exit the survey at any time. After the survey
closed, regardless if it was complete or incomplete, access was denied
to any further users with the same internet protocol (IP) address and
the domain entry point rotated such that if a user were to attempt to
access the survey again, share the link, or enter via the same address
using an alternative IP address, the survey would not deploy. This
effectively meant that a user could participate only once in the survey.

Additionally, users who indicated they were under the age of 16 were
exited from the survey without creating a record, furthermore, these
users were unable to navigate back to the ``age select'' screen. The
personal identifier information from each user that participated in the
survey was automatically scrubbed and replaced by a unique ID.
Respondents were drawn exclusively from the province of Ontario, as per
their devices meta-data.

\hypertarget{sec-socio-demographic-factors}{%
\subsection{Survey responses}\label{sec-socio-demographic-factors}}

Socio-economic information selected from the survey answers included age
group and income brackets, and race/ethnicity. The levels of each
socio-economic factor used for analysis appear in
Table~\ref{tbl-covariates}.

\hypertarget{tbl-covariates}{}
\begin{longtable}[]{@{}
  >{\raggedright\arraybackslash}p{(\columnwidth - 2\tabcolsep) * \real{0.3014}}
  >{\raggedright\arraybackslash}p{(\columnwidth - 2\tabcolsep) * \real{0.6986}}@{}}
\caption{\label{tbl-covariates}Selected socio-economic factors from the
survey}\tabularnewline
\toprule()
\begin{minipage}[b]{\linewidth}\raggedright
Variable
\end{minipage} & \begin{minipage}[b]{\linewidth}\raggedright
Levels
\end{minipage} \\
\midrule()
\endfirsthead
\toprule()
\begin{minipage}[b]{\linewidth}\raggedright
Variable
\end{minipage} & \begin{minipage}[b]{\linewidth}\raggedright
Levels
\end{minipage} \\
\midrule()
\endhead
Age group & 16-34,35-54,55 and over \\
Income bracket (CAD) & under 25,000, 25,000-59,999, 60,000 and above \\
Race/ethnicity & Arab/Middle Eastern, Black, East Asian/Pacific
Islander, Indigenous, Latin American, Mixed, South Asian, White
Caucasian, Other \\
\bottomrule()
\end{longtable}

Furthermore, the information on vaccination status was provided by
survey participants who answered the question ``Have you received the
first dose of the COVID vaccine?'', with possible answers ``yes'' and
``no''.

\hypertarget{data-cleaning}{%
\subsection{Data cleaning}\label{data-cleaning}}

The original dataset contained 39,029 entries (where each entry
corresponded to a set of answers provided by a unique respondent).
Following a preliminary analysis, the dataset was cleaned in order to
only contain the socio-economic information provided in
Table~\ref{tbl-covariates} and vaccination status. The cleaning process
also included removing outliers that were identified during the
preliminary analyses, and processing the geographical information in the
survey (city where the survey was responded) in order to match each city
to its correspondent Health Region.

\hypertarget{sec-geographical-location}{%
\subsubsection{Geographical location}\label{sec-geographical-location}}

For each survey participant certain data was automatically captured,
including the nearest municipality (city). This resulted in a total of
578 different municipalities within the clean dataset. Due to our
interest in analyzing the differences between Health Regions, we
assigned the city of each entry to its correspondent Health Region
following a multi-step process. Briefly, we used Local Health Integrated
Networks (LHINs) to assign a Health Region to each entry in the survey.
LHINs were the geographical divisions for health used by Ontario before
the adoption of the Health Regions; because of the lack of a publicly
available list of all municipalities within each Health Region, we used
a dataset of long-term care homes and LHINs to match each city to LHIN,
followed by matching each LHIN to a Health Region following the
information provided on the Ontario Health Website, where the list of
LHINs and corresponding Health Regions is available. In the case of
municipalities that did not appear in the long-term dataset, we manually
searched each city in the LHINs websites in order to provide
geographical information. The raw dataset, clean dataset, details of the
data cleaning process, and the addition of Health Region and LHIN
information are described in detail in the GitHub repository for this
paper, which can be found at
\url{https://github.com/aimundo/Fields_COVID-19/}.

Following an assessment of the number of entries corresponding to each
Health Region in the final dataset, only 107 observations (4.3\% of the
total) corresponded to cities located in the North West and North East
Health Regions. Due to the low number of entries, we omitted these
Health Regions from further analyses. Therefore, the total number of
unique entries used for analysis was 3,551 which included the East,
Central, Toronto, and West Health Regions.

\hypertarget{corrections}{%
\subsection{Corrections}\label{corrections}}

We identified differences between the proportions of all the
socio-economic factors included in the analysis
(Table~\ref{tbl-covariates}) and the 2016 Canada Census data for
Ontario. Additionally, because the Census divisions do not match the
exact boundaries of the Health Regions, we also obtained population
estimates for each Health Region from the Ontario Health website in
order to correct for the population totals. We used an iterative
proportional fitting procedure
(\emph{raking})\textsuperscript{\protect\hyperlink{ref-deming1940}{21}}
to correct for socio-economic factors and Health Region populations
using the R \texttt{survey} package. Details about the correction can be
found in the Appendix.

\hypertarget{statistical-analyses}{%
\subsection{Statistical analyses}\label{statistical-analyses}}

We used a logistic regression model to estimate the probability of
vaccination depending on the socio-economic factors described in
Section~\ref{sec-socio-demographic-factors}, the Health Regions from
Section~\ref{sec-geographical-location}, and the interactions between
Race and Health Region, and Race and income as previous studies have
shown that socio-economic factors and their interactions are significant
predictors of intent of vaccination and vaccination
status\textsuperscript{\protect\hyperlink{ref-nguyen2022}{22}--\protect\hyperlink{ref-cnat2022a}{24}}.
The model appears in Equation~\ref{eq-model3},

\begin{equation}\protect\hypertarget{eq-model3}{}{
\begin{aligned}
\log \left( \frac{p\textrm{(vac)}}{1-p\textrm{(vac)}} \right) = \beta_0+ \beta_{1}\textrm{(Age group)} +\beta_{2} \textrm{ Race} + \beta_3 \textrm{ Health Region} + \beta_4 \textrm{ Income}+\\ \beta_5\textrm{(Health Region} \times \textrm{Race)} + \beta_6 \textrm{ (Income} \times \textrm{Race)}
\end{aligned}
}\label{eq-model3}\end{equation}

Where \(p\textrm{(vac)}\) indicates the probability of having received
the first dose of a COVID-19 vaccine, \(\beta_0\) indicates the
population intercept, and \(\beta_1...\beta_6\) indicate the
coefficients for each term. The the model was fitted using the function
\texttt{svyglm} from the \texttt{survey} R package in order to
incorporate the correction in sampling probability obtained from raking.

All analyses were conducted in RStudio (2022.12.0 Build 353), using R
4.2.2 and the packages
\texttt{survey}\textsuperscript{\protect\hyperlink{ref-lumley2011}{25}},\texttt{tidyverse}\textsuperscript{\protect\hyperlink{ref-wickham2019}{26}},
and \texttt{quarto}\textsuperscript{\protect\hyperlink{ref-quarto}{27}}.

\hypertarget{results}{%
\section{Results}\label{results}}

\hypertarget{survey-results}{%
\subsection{Survey Results}\label{survey-results}}

Table~\ref{tbl-descriptive-stats} shows the descriptive statistics
(uncorrected) from the Fields COVID-19 survey data for vaccination
status and each of the covariates analyzed. The total number of entries
from the survey in the dataset after cleaning was 3,551. Overall, 26.9\%
of survey respondents (958) reported not having received the first dose
of the vaccine, whereas 73.1\% (2,593) reported having received it.
Within each socio-economic factor, respondents who reported living in a
household with an income under CAD 25,000 represented 37\% of the total
number of entries, those within the CAD 25,000-59,999 income bracket
represented 25\% of the total sample, and those with an income above CAD
60,000 represented 38 \% of the sample; across all income brackets, the
percentage of individuals that reported having received a first dose of
the vaccine was consistent, above 69\%.

Within the age groups of survey respondents, the age group between 16-34
years had the highest representation in the survey responses (1,521,
42.8\% of all responses). Within this age bracket, 73\% of respondents
indicated having received the vaccine, whereas the lowest vaccination
rate was in the bracket of those 55 years of age and above, with a total
of 72\%. The Health Region with highest representation in the survey was
Toronto, accounting for 1,324 entries (37.2\%), with a vaccination rate
of 72\%. Regarding race/ethnicity, individuals that identified as
White/Caucasian represented 1313 (37\%) of all entries and had the
highest vaccination uptake with 82\% of them indicating to have received
the COVID-19 vaccine. On the other hand, the ethnic group with the
lowest number of entries in the survey was Latin American, with a total
of 180, or 5\% of all entries. Vaccination rates across all minority
groups were below the value reported by White/Caucasians, with the
lowest vaccination rate (60\%) being reported by individuals that
identified as Indigenous.

\hypertarget{tbl-descriptive-stats}{}
\begin{table}
\caption{\label{tbl-descriptive-stats}Descriptive Statistics of the Fields COVID-19 Survey (by Vaccination
Status) }\tabularnewline

\centering\begingroup\fontsize{9}{11}\selectfont

\begin{tabular}{lcc}
\toprule
\textbf{Variable} & \textbf{no}, N = 958 & \textbf{yes}, N = 2,593\\
\midrule
\textbf{Income} &  & \\
\hspace{1em}60000\_and\_above & 305 (23\%) & 1,049 (77\%)\\
\hspace{1em}25000\_59999 & 253 (28\%) & 636 (72\%)\\
\hspace{1em}under\_25000 & 400 (31\%) & 908 (69\%)\\
\textbf{Age Group} &  & \\
\hspace{1em}16\_34 & 409 (27\%) & 1,112 (73\%)\\
\hspace{1em}35\_54 & 252 (26\%) & 712 (74\%)\\
\hspace{1em}55\_and\_over & 297 (28\%) & 769 (72\%)\\
\textbf{Health Region} &  & \\
\hspace{1em}Toronto & 371 (28\%) & 953 (72\%)\\
\hspace{1em}Central & 224 (28\%) & 581 (72\%)\\
\hspace{1em}East & 135 (23\%) & 448 (77\%)\\
\hspace{1em}West & 228 (27\%) & 611 (73\%)\\
\textbf{Race} &  & \\
\hspace{1em}white\_caucasian & 233 (18\%) & 1,080 (82\%)\\
\hspace{1em}arab\_middle\_eastern & 76 (36\%) & 138 (64\%)\\
\hspace{1em}black & 114 (38\%) & 184 (62\%)\\
\hspace{1em}east\_asian\_pacific\_islander & 69 (23\%) & 234 (77\%)\\
\hspace{1em}indigenous & 76 (40\%) & 115 (60\%)\\
\hspace{1em}latin\_american & 69 (38\%) & 111 (62\%)\\
\hspace{1em}mixed & 105 (34\%) & 205 (66\%)\\
\hspace{1em}other & 128 (35\%) & 239 (65\%)\\
\hspace{1em}south\_asian & 88 (23\%) & 287 (77\%)\\
\bottomrule
\multicolumn{3}{l}{\rule{0pt}{1em}\textsuperscript{1} n (\%)}\\
\end{tabular}
\endgroup{}
\end{table}

\hypertarget{multivariate-regression}{%
\subsection{Multivariate Regression}\label{multivariate-regression}}

Table~\ref{tbl-model} shows the results of the logistic regression on
vaccination status using socio-economic factors (age group, income,
race), geographical areas (Health Regions) and the interactions between
income and race and Health Region and race. The reference groups were
set as follows: 16 to 34 years (age group), White Caucasian (Race),
Toronto (Health Region), CAD 60,000 and over (Income). There were no
statistically significant differences in vaccination rates within the
age groups from the survey, but significant odds ratios were estimated
for other covariates. Within household income brackets, individuals with
an income under CAD 25,000 or between CAD 25,000-59,999 had
significantly lower odds of vaccination than those with an income above
CAD 60,000 (ORs=0.37 and 0.59, respectively). Within Race/Ethnicity,
individuals who identified as Arab/Middle Eastern, Black, or Latin
American, had significantly lower odds of vaccination than those in the
White/Caucasian group (ORs=0.31, 0.32, 0.28, and
\emph{p}=0.004,\textless0.001 and 0.004, respectively); additionally,
those individuals in the Other Race/Ethnicity group (a group that
included Southeast Asian, Filipino, West Asian, and Minorities Not
Identified Elsewhere) had even lower odds of vaccination than the other
minority groups (OR=0.22, \emph{p}\textless0.001). Regarding Health
Regions, individuals that reported living in the West Health Region
(which comprises the regions of Waterloo and Niagara, the counties of
Wellington, Essex, and Lambton, and the cities of Hamilton, Haldimand,
Brant, and Chatham-Kent) had significantly higher odds of vaccination
than those in the Health Region of Toronto (OR=1.55, \emph{p}=0.029).

Moreover, statistically-significant odd ratios were determined in the
case of the interaction of income and race; specifically, for
individuals with a household income below CAD 25,000 who identified as
Arab/Middle Eastern (OR=3.05, \emph{p}=0.013), Black (OR=3.19,
\emph{p}=0.004), Latin American (OR=2.80, \emph{p}=0.041), or that
belonged to other minority groups (OR=4.59, \emph{p}\textless0.001).
Within the CAD 25,000-59,999 income bracket, individuals who identified
as belonging to other racial minority groups had significantly higher
odds of vaccination (OR=6.93, \emph{p}\textless0.001).

For the interaction of Health Region and race, significant odds of
vaccination were identified for Black individuals in the Central Health
Region, which comprises the region of York, counties of Dufferin and
Simcoe and the district of Muskoka (OR=0.44, \emph{p}=0.046), and in
individuals that identified as part of other racial minorities or South
Asian that lived in the West Health Region (ORs=0.41, \emph{p}=0.032 and
\emph{p}=0.037, respectively).

\footnotesize
\renewcommand{\arraystretch}{0.5}

\hypertarget{tbl-model}{}
\begin{longtable}{lccc}
\caption{\label{tbl-model}Multiple Regression Analysis-Predictors of Vaccination Status }\tabularnewline

\toprule
\textbf{Characteristic} & \textbf{OR} & \textbf{95\% CI} & \textbf{p-value}\\
\midrule
\endfirsthead
\multicolumn{4}{@{}l}{\textit{(continued)}}\\
\toprule
\textbf{Characteristic} & \textbf{OR} & \textbf{95\% CI} & \textbf{p-value}\\
\midrule
\endhead

\endfoot
\bottomrule
\endlastfoot
\textbf{Age Group} &  &  & \\
\hspace{1em}16\_34 & — & — & \\
\hspace{1em}35\_54 & 0.90 & 0.67, 1.21 & 0.5\\
\hspace{1em}55\_and\_over & 0.99 & 0.74, 1.32 & >0.9\\
\textbf{Income} &  &  & \\
\hspace{1em}60000\_and\_above & — & — & \\
\hspace{1em}25000\_59999 & 0.59 & 0.39, 0.89 & 0.011\\
\hspace{1em}under\_25000 & 0.37 & 0.25, 0.56 & <0.001\\
\textbf{Race} &  &  & \\
\hspace{1em}white\_caucasian & — & — & \\
\hspace{1em}arab\_middle\_eastern & 0.31 & 0.14, 0.69 & 0.004\\
\hspace{1em}black & 0.32 & 0.17, 0.60 & <0.001\\
\hspace{1em}east\_asian\_pacific\_islander & 1.15 & 0.50, 2.66 & 0.7\\
\hspace{1em}indigenous & 0.44 & 0.19, 1.02 & 0.056\\
\hspace{1em}latin\_american & 0.28 & 0.11, 0.67 & 0.004\\
\hspace{1em}mixed & 0.64 & 0.25, 1.65 & 0.4\\
\hspace{1em}other & 0.22 & 0.12, 0.41 & <0.001\\
\hspace{1em}south\_asian & 0.91 & 0.49, 1.69 & 0.8\\
\textbf{Health Region} &  &  & \\
\hspace{1em}Toronto & — & — & \\
\hspace{1em}Central & 1.47 & 0.92, 2.35 & 0.11\\
\hspace{1em}East & 1.42 & 0.90, 2.23 & 0.13\\
\hspace{1em}West & 1.55 & 1.05, 2.30 & 0.029\\
\textbf{Income * Race} &  &  & \\
\hspace{1em}25000\_59999 * arab\_middle\_eastern & 1.79 & 0.67, 4.83 & 0.2\\
\hspace{1em}under\_25000 * arab\_middle\_eastern & 3.05 & 1.26, 7.39 & 0.013\\
\hspace{1em}25000\_59999 * black & 1.34 & 0.59, 3.05 & 0.5\\
\hspace{1em}under\_25000 * black & 3.19 & 1.45, 6.99 & 0.004\\
\hspace{1em}25000\_59999 * east\_asian\_pacific\_islander & 0.42 & 0.17, 1.05 & 0.062\\
\hspace{1em}under\_25000 * east\_asian\_pacific\_islander & 1.16 & 0.47, 2.86 & 0.8\\
\hspace{1em}25000\_59999 * indigenous & 1.36 & 0.48, 3.89 & 0.6\\
\hspace{1em}under\_25000 * indigenous & 1.45 & 0.55, 3.80 & 0.5\\
\hspace{1em}25000\_59999 * latin\_american & 1.24 & 0.45, 3.43 & 0.7\\
\hspace{1em}under\_25000 * latin\_american & 2.80 & 1.04, 7.51 & 0.041\\
\hspace{1em}25000\_59999 * mixed & 0.85 & 0.32, 2.26 & 0.7\\
\hspace{1em}under\_25000 * mixed & 1.10 & 0.37, 3.27 & 0.9\\
\hspace{1em}25000\_59999 * other & 6.93 & 2.65, 18.1 & <0.001\\
\hspace{1em}under\_25000 * other & 4.59 & 2.33, 9.05 & <0.001\\
\hspace{1em}25000\_59999 * south\_asian & 1.20 & 0.51, 2.85 & 0.7\\
\hspace{1em}under\_25000 * south\_asian & 2.00 & 0.93, 4.30 & 0.077\\
\textbf{Race * Health Region} &  &  & \\
\hspace{1em}arab\_middle\_eastern * Central & 0.66 & 0.26, 1.70 & 0.4\\
\hspace{1em}black * Central & 0.44 & 0.19, 0.98 & 0.046\\
\hspace{1em}east\_asian\_pacific\_islander * Central & 0.98 & 0.38, 2.53 & >0.9\\
\hspace{1em}indigenous * Central & 0.63 & 0.22, 1.79 & 0.4\\
\hspace{1em}latin\_american * Central & 0.67 & 0.23, 1.96 & 0.5\\
\hspace{1em}mixed * Central & 0.73 & 0.24, 2.22 & 0.6\\
\hspace{1em}other * Central & 0.80 & 0.36, 1.78 & 0.6\\
\hspace{1em}south\_asian * Central & 0.54 & 0.25, 1.20 & 0.13\\
\hspace{1em}arab\_middle\_eastern * East & 0.43 & 0.13, 1.45 & 0.2\\
\hspace{1em}black * East & 0.83 & 0.34, 2.04 & 0.7\\
\hspace{1em}east\_asian\_pacific\_islander * East & 0.86 & 0.29, 2.56 & 0.8\\
\hspace{1em}indigenous * East & 0.69 & 0.23, 2.08 & 0.5\\
\hspace{1em}latin\_american * East & 1.03 & 0.32, 3.34 & >0.9\\
\hspace{1em}mixed * East & 0.91 & 0.28, 3.03 & 0.9\\
\hspace{1em}other * East & 1.05 & 0.39, 2.83 & >0.9\\
\hspace{1em}south\_asian * East & 0.52 & 0.19, 1.45 & 0.2\\
\hspace{1em}arab\_middle\_eastern * West & 1.00 & 0.37, 2.73 & >0.9\\
\hspace{1em}black * West & 0.76 & 0.32, 1.80 & 0.5\\
\hspace{1em}east\_asian\_pacific\_islander * West & 0.52 & 0.20, 1.34 & 0.2\\
\hspace{1em}indigenous * West & 0.39 & 0.14, 1.09 & 0.073\\
\hspace{1em}latin\_american * West & 0.94 & 0.32, 2.72 & >0.9\\
\hspace{1em}mixed * West & 0.37 & 0.12, 1.16 & 0.089\\
\hspace{1em}other * West & 0.41 & 0.18, 0.93 & 0.032\\
\hspace{1em}south\_asian * West & 0.41 & 0.18, 0.95 & 0.037\\*
\multicolumn{4}{l}{\rule{0pt}{1em}\textsuperscript{1} OR = Odds Ratio, CI = Confidence Interval}\\
\end{longtable}

\normalsize

\hypertarget{discussion}{%
\section{Discussion}\label{discussion}}

The rapid development of COVID-19 vaccines has been considered as a
major achievement of modern
medicine\textsuperscript{\protect\hyperlink{ref-davis2022}{28}}. Vaccine
availability towards the end of 2020 in certain countries made some
believe that they would be a determinant factor in a rapid ending of the
pandemic\textsuperscript{\protect\hyperlink{ref-thelancet2021}{29}}.
However, despite previous successful vaccination campaigns that were
crucial to control diseases such as smallpox and
polio\textsuperscript{\protect\hyperlink{ref-kayser2021}{30}},
vaccination efforts in the case of COVID-19 have faced multiple
challenges that have complicated the achievement of global immunity.

Among the different challenges faced by COVID-19 vaccination efforts are
the development of new variants due to inadequate public health
measures\textsuperscript{\protect\hyperlink{ref-li2021}{31}} and
inequity in vaccine access between low and high income
countries\textsuperscript{\protect\hyperlink{ref-tamey2022}{32}}.
However, it is also well established that even in the case of high
income countries that have had ample access to vaccines since 2020, such
as the US, the UK, and Canada, there have been challenges in vaccination
efforts due to differences in vaccine uptake among different segments of
the population. More specifically, lower vaccine uptake has been
associated with socio-economic factors such as race (i.e., identifying
as Black, Asian, Indigenous) and household income (typically within
lower income
brackets)\textsuperscript{\protect\hyperlink{ref-willis2021}{33}--\protect\hyperlink{ref-khubchandani2021}{36}}.
Reasons given for this association have included medical mistrust due to
systemic medical racism, mistrust in vaccines, and the influence of
conspiracy
theories\textsuperscript{\protect\hyperlink{ref-willis2021}{33},\protect\hyperlink{ref-stoler2021}{35},\protect\hyperlink{ref-bogart2021}{37}--\protect\hyperlink{ref-freeman2020}{39}}.

In addition, vaccine uptake is influenced by geography, as shown by
different studies that have identified intra-regional differences in
vaccine
uptake\textsuperscript{\protect\hyperlink{ref-mollalo2021}{12},\protect\hyperlink{ref-pallathadka2022}{40},\protect\hyperlink{ref-huang2022}{41}}.
However, in the case of Canada, studies that have analyzed spatial
differences in vaccination have been focused in country-wide or
province-wide
estimates\textsuperscript{\protect\hyperlink{ref-guay2022}{8},\protect\hyperlink{ref-lavoie2022}{42}}.
Therefore, we explored spatial and socio-economic determinants of
vaccination status in the province of Ontario. This province is of
particular interest as it has seen recently major structural health
changes with the dissolution of the Local Health Integrated Network
(LHIN) system and the incorporation of regions covered by LHINs into
larger Health
Regions\textsuperscript{\protect\hyperlink{ref-dong2022}{43}}. Because
the idea behind the change aimed to reduce the inequalities in
healthcare that were identified under the LHIN
model\textsuperscript{\protect\hyperlink{ref-tsasis2012}{44}}, examining
differences in vaccination between the Health Regions can provide
decision-makers with insight regarding intra-provincial health
disparities that may need to be addressed in future vaccination or
public health campaigns.

Our results indicate that across the most densely populated Health
Regions of Ontario, almost three quarters of the surveyed individuals
reported having received the first dose of the COVID-19 vaccine
(Table~\ref{tbl-descriptive-stats}), and that there were no significant
differences in vaccination odds among the age groups considered in the
survey. This result is consistent with overall vaccination rates
reported for Canada, which have been relatively higher when compared to
other high income
countries\textsuperscript{\protect\hyperlink{ref-dube2022}{45}}, with
vaccination uptake rates across different age groups presented in other
studies\textsuperscript{\protect\hyperlink{ref-guay2022}{8},\protect\hyperlink{ref-macdonald2021}{46}},
and with the vaccination information provided by Public Health Ontario,
which shows that for the period where the Fields survey ran (Sept 30,
2021-Jan 17, 2022) there was a minimum of 80\% of first dose vaccination
coverage among all the age groups considered in the
survey\textsuperscript{\protect\hyperlink{ref-ontario-covid}{47}}.

However, we identified intra-provincial differences in vaccination based
on socio-economic and geographical factors. First, our results show
significant differences in vaccination odds in individuals with a
household income below CAD 60,000 and in individuals belonging to
visible minority groups. Those who identified as Black, Latin American,
or belonging to a minority group not included in the survey (Southeast
Asian, Filipino, West Asian, and Minority not identified elsewhere) had
vaccination odds below 33\% when compared to individuals that identified
as White/Caucasian (Table~\ref{tbl-model}). These results are consistent
with other studies that have shown lower vaccination rates in
individuals that identify as part of a racial minority, or that have a
low household
income\textsuperscript{\protect\hyperlink{ref-guay2022}{8}--\protect\hyperlink{ref-hussain2022}{10},\protect\hyperlink{ref-carter2022}{48}}.

In this study, we also decided to explore the interactions between
income and race and race and Health Region, as it is known that many
individuals within racial minority groups perform tend to occupy certain
types of occupations that fall within income brackets that have been
shown to be associated with differences in vaccination uptake. In other
words, we decided to explore if there were differences in vaccination
within racial groups in certain income brackets and in certain the
Health Regions. In this regard, it is interesting to note that although
overall self-reported vaccination rates were found to be statistically
significantly lower in various racial minority groups when compared to
White/Caucasian individuals (Table~\ref{tbl-model}), the change in odds
of vaccination within certain racial groups and income strata was
actually positive, in contrast to the White/Caucasian group, for which
vaccination odds decreased in lower income brackets (when compared to
the CAD 60,000 and over bracket, Supplementary Figure A-3). More
specifically, the change in odds of vaccination increased in individuals
who identified as Arab/Middle Eastern, Black, Latin American, or
belonging to other minority groups with a household income below CAD
25,000, which was also true for individuals in other racial minority
groups with an income between CAD 25,000-59,999 (Table~\ref{tbl-model},
Supplementary Figure A-3).

This result is likely due to the fact that individuals that belong
racial minority groups tend to perform occupations that have been deemed
as ``essential'' in the context of the
pandemic\textsuperscript{\protect\hyperlink{ref-hawkins2020}{49},\protect\hyperlink{ref-ct2021}{50}},
which include occupations such as grocery store workers, gas station
workers, warehouse and distribution workers, and manufacturing workers,
all being occupations for which an income within the significant
brackets is to be expected. In the case of Ontario, essential workers
had priority for COVID-19
vaccination\textsuperscript{\protect\hyperlink{ref-mishra2021}{51}},
which would explain the higher odds of vaccination for these individuals
in certain income brackets, in contrast to the lower odds of vaccination
for the same type of individuals with higher household income. In other
words, it is possible that the type of occupation played an important
role in increasing the odds of vaccination in these racial minority
groups.

Additionally, significant higher vaccination odds were identified in the
West Health Region when compared to the Health Region of Toronto
(Table~\ref{tbl-model}). The West Health Region comprises the regions of
Waterloo and Niagara, the counties of Wellington, Essex and Lambton, and
the cities of Hamilton, Haldimand, Brant, and Chatham-Kent. In this
case, a possible rationale for the results is the fact that in the
survey, about 47\% of the entries for this Health Region corresponded to
White/Caucasian individuals, who reported an overall 83\% vaccination
rate (Supplementary Table A-6). However, the interaction effect of
Health Region and race was also significant in the case of individuals
identifying as South Asian or other minorities not included in the
survey Table~\ref{tbl-model}. In this case, the results of the
interaction term in the model indicate that the odds of vaccination for
those within the South Asian and Other minority groups in the West
Region decreased when compared to the other Health Regions
(Supplementary Figure A-4).

According to Ontario Health, 13.2\% of the population in the West Health
Region identifies as a visible minority, whereas 2.5\% identifies as
Indigenous\textsuperscript{\protect\hyperlink{ref-ontariohealth}{52}}.
In the case of this analysis, the estimated lower odds are likely to be
explained from a socio-economic perspective. In fact, 50\% of the
answers from this region in the survey came from the former LHINs of
Hamilton Niagara Haldimand Brant, and Erie St.~Clair, both which are
among the regions of Ontario with the highest proportion of their
population (more than 20\%) in the lowest income
quintile\textsuperscript{\protect\hyperlink{ref-buajitti2018}{53}}
(Supplementary Table A-7). Therefore, this result partly reinforces the
well-known existing association between low vaccination rates and
income, but it additionally indicates that there were intra-regional
differences in vaccination. Interestingly, a disproportionate number of
COVID-19 cases and low vaccination rate (under 50\%) have been
previously reported in the South Asian community of
Ontario\textsuperscript{\protect\hyperlink{ref-anand2022}{54}}; in this
regard, our result provides additional context by showing that within
the South Asian community, there were differences in vaccination uptake
across Ontario. Moreover, because significant lower odds of vaccination
were also identified other minority groups, this provides a rationale
for future studies that explore how vaccination uptake varies across
different minority groups within Ontario and other Canadian provinces.

There are some limitations to the present study. First, the data
collection design, which allowed respondents to withdraw from the survey
at any point, resulted in a high number of unique entries in the survey
with multiple missing answers. Because we focused on entries that had
complete observations in the covariates of interest for our analysis, it
is possible that some information was not considered by excluding
observations that had information in other variables (such as work from
home, or number of persons in the household). However, we attempted to
minimize this possibility by correcting the dataset using information
from the Census. More granular corrections, which for example could be
based on demographic information by municipality, could be used in the
future to obtain a more accurate approximation to the population totals
of the province. Additionally, the results in this study are based on
self-reported data, where the risk of bias exist. Despite this, because
in the context of COVID-19 it has been shown that good agreement exists
between self-reported and documented vaccination
status\textsuperscript{\protect\hyperlink{ref-stephenson2022}{55}}, the
effect of self-reported bias is likely to not be significant in this
case.

Finally, it is likely that there have been differences in vaccination
across the province as more doses of the vaccine were administered and
as successive variants emerged. Because this study focused only on
vaccination status regarding the first dose of the vaccine within a
relatively short time window, it can only provide a snapshot of the
societal dynamics behind the pandemic. Nonetheless, the results
presented here can serve as a starting point to motivate future
longitudinal research that aims to quantify geographical differences
within vulnerable segments of the population, and that can be used to
inform the development of public health policies within the province of
Ontario or across other provinces that aim to minimize disparities in
health access.

\hypertarget{conclusion}{%
\section{Conclusion}\label{conclusion}}

This study explored differences in COVID-19 vaccination across the
province of Ontario between late 2021 and early 2022 by taking into
consideration socio-economic factors, such as income and race, their
interactions, and the Health Regions within the province. Our results
show that, during the period analyzed, significant differences in
vaccination existed across different visible minority groups, income
brackets, and Health Regions, showing intra-provincial disparities in
vaccine uptake. As the COVID-19 continues around the world, it important
that future public policies take into consideration how to adequately
reach individuals within minority groups that live across geographical
areas where less probabilities of being vaccinated are likely. At the
moment, this is an ongoing issue that needs to be addressed to ensure a
more homogeneous outcome from the pandemic.

\hypertarget{references}{%
\section{References}\label{references}}

\hypertarget{refs}{}
\begin{CSLReferences}{0}{0}
\leavevmode\vadjust pre{\hypertarget{ref-WHO-Covid}{}}%
\CSLLeftMargin{1. }%
\CSLRightInline{{World Health Organization Coronavirus (COVID-19)
Dashboard}. Accessed November 27, 2022. \url{https://covid19.who.int/}}

\leavevmode\vadjust pre{\hypertarget{ref-tanne2020}{}}%
\CSLLeftMargin{2. }%
\CSLRightInline{Tanne JH. Covid-19: {FDA} panel votes to authorise
pfizer {BioNTech} vaccine. \emph{{BMJ}}. Published online December
2020:m4799.
doi:\href{https://doi.org/10.1136/bmj.m4799}{10.1136/bmj.m4799}}

\leavevmode\vadjust pre{\hypertarget{ref-bogoch2022}{}}%
\CSLLeftMargin{3. }%
\CSLRightInline{Bogoch II, Halani S. {COVID}-19 vaccines: A geographic,
social and policy view of vaccination efforts in ontario, canada.
\emph{Cambridge Journal of Regions, Economy and Society}. Published
online November 2022.
doi:\href{https://doi.org/10.1093/cjres/rsac043}{10.1093/cjres/rsac043}}

\leavevmode\vadjust pre{\hypertarget{ref-watson2022}{}}%
\CSLLeftMargin{4. }%
\CSLRightInline{Watson OJ, Barnsley G, Toor J, Hogan AB, Winskill P,
Ghani AC. Global impact of the first year of {COVID}-19 vaccination: A
mathematical modelling study. \emph{The Lancet Infectious Diseases}.
2022;22(9):1293-1302.
doi:\href{https://doi.org/10.1016/s1473-3099(22)00320-6}{10.1016/s1473-3099(22)00320-6}}

\leavevmode\vadjust pre{\hypertarget{ref-gerretsen2021}{}}%
\CSLLeftMargin{5. }%
\CSLRightInline{Gerretsen P, Kim J, Caravaggio F, et al. Individual
determinants of {COVID}-19 vaccine hesitancy. Inbaraj LR, ed.
\emph{{PLOS} {ONE}}. 2021;16(11):e0258462.
doi:\href{https://doi.org/10.1371/journal.pone.0258462}{10.1371/journal.pone.0258462}}

\leavevmode\vadjust pre{\hypertarget{ref-nafilyan2021}{}}%
\CSLLeftMargin{6. }%
\CSLRightInline{Nafilyan V, Dolby T, Razieh C, et al. Sociodemographic
inequality in {COVID}-19 vaccination coverage among elderly adults in
england: A national linked data study. \emph{{BMJ} Open}.
2021;11(7):e053402.
doi:\href{https://doi.org/10.1136/bmjopen-2021-053402}{10.1136/bmjopen-2021-053402}}

\leavevmode\vadjust pre{\hypertarget{ref-malik2020}{}}%
\CSLLeftMargin{7. }%
\CSLRightInline{Malik AA, McFadden SM, Elharake J, Omer SB. Determinants
of {COVID}-19 vaccine acceptance in the {US}.
\emph{{EClinicalMedicine}}. 2020;26:100495.
doi:\href{https://doi.org/10.1016/j.eclinm.2020.100495}{10.1016/j.eclinm.2020.100495}}

\leavevmode\vadjust pre{\hypertarget{ref-guay2022}{}}%
\CSLLeftMargin{8. }%
\CSLRightInline{Guay M, Maquiling A, Chen R, et al. Measuring
inequalities in {COVID}-19 vaccination uptake and intent: Results from
the canadian community health survey 2021. \emph{{BMC} Public Health}.
2022;22(1).
doi:\href{https://doi.org/10.1186/s12889-022-14090-z}{10.1186/s12889-022-14090-z}}

\leavevmode\vadjust pre{\hypertarget{ref-muhajarine2021}{}}%
\CSLLeftMargin{9. }%
\CSLRightInline{Muhajarine N, Adeyinka DA, McCutcheon J, Green KL,
Fahlman M, Kallio N. {COVID}-19 vaccine hesitancy and refusal and
associated factors in an adult population in saskatchewan, canada:
Evidence from predictive modelling. Gesser-Edelsburg A, ed. \emph{{PLOS}
{ONE}}. 2021;16(11):e0259513.
doi:\href{https://doi.org/10.1371/journal.pone.0259513}{10.1371/journal.pone.0259513}}

\leavevmode\vadjust pre{\hypertarget{ref-hussain2022}{}}%
\CSLLeftMargin{10. }%
\CSLRightInline{Hussain B, Latif A, Timmons S, Nkhoma K, Nellums LB.
Overcoming {COVID}-19 vaccine hesitancy among ethnic minorities: A
systematic review of {UK} studies. \emph{Vaccine}.
2022;40(25):3413-3432.
doi:\href{https://doi.org/10.1016/j.vaccine.2022.04.030}{10.1016/j.vaccine.2022.04.030}}

\leavevmode\vadjust pre{\hypertarget{ref-nguyen2021}{}}%
\CSLLeftMargin{11. }%
\CSLRightInline{Nguyen KH, Nguyen K, Corlin L, Allen JD, Chung M.
Changes in {COVID}-19 vaccination receipt and intention to vaccinate by
socioeconomic characteristics and geographic area, united states,
january 6 {\textendash} march 29, 2021. \emph{Annals of Medicine}.
2021;53(1):1419-1428.
doi:\href{https://doi.org/10.1080/07853890.2021.1957998}{10.1080/07853890.2021.1957998}}

\leavevmode\vadjust pre{\hypertarget{ref-mollalo2021}{}}%
\CSLLeftMargin{12. }%
\CSLRightInline{Mollalo A, Tatar M. Spatial modeling of {COVID}-19
vaccine hesitancy in the united states. \emph{International Journal of
Environmental Research and Public Health}. 2021;18(18):9488.
doi:\href{https://doi.org/10.3390/ijerph18189488}{10.3390/ijerph18189488}}

\leavevmode\vadjust pre{\hypertarget{ref-yang2022}{}}%
\CSLLeftMargin{13. }%
\CSLRightInline{Yang TC, Matthews SA, Sun F. Multiscale dimensions of
spatial process: {COVID}-19 fully vaccinated rates in u.s. counties.
\emph{American Journal of Preventive Medicine}. 2022;63(6):954-961.
doi:\href{https://doi.org/10.1016/j.amepre.2022.06.006}{10.1016/j.amepre.2022.06.006}}

\leavevmode\vadjust pre{\hypertarget{ref-tiu2022}{}}%
\CSLLeftMargin{14. }%
\CSLRightInline{Tiu A, Susswein Z, Merritt A, Bansal S. Characterizing
the spatiotemporal heterogeneity of the {COVID}-19 vaccination
landscape. \emph{American Journal of Epidemiology}.
2022;191(10):1792-1802.
doi:\href{https://doi.org/10.1093/aje/kwac080}{10.1093/aje/kwac080}}

\leavevmode\vadjust pre{\hypertarget{ref-bhuiyan2022}{}}%
\CSLLeftMargin{15. }%
\CSLRightInline{Bhuiyan MAN, Davis TC, Arnold CL, et al. Using the
social vulnerability index to assess {COVID}-19 vaccine uptake in
louisiana. \emph{{GeoJournal}}. Published online December 2022.
doi:\href{https://doi.org/10.1007/s10708-022-10802-5}{10.1007/s10708-022-10802-5}}

\leavevmode\vadjust pre{\hypertarget{ref-wood2022}{}}%
\CSLLeftMargin{16. }%
\CSLRightInline{Wood AJ, MacKintosh AM, Stead M, Kao RR. Predicting
future spatial patterns in {COVID}-19 booster vaccine uptake. Published
online September 2022.
doi:\href{https://doi.org/10.1101/2022.08.30.22279415}{10.1101/2022.08.30.22279415}}

\leavevmode\vadjust pre{\hypertarget{ref-choi2021}{}}%
\CSLLeftMargin{17. }%
\CSLRightInline{Choi KH, Denice PA, Ramaj S. Vaccine and {COVID}-19
trajectories. \emph{Socius: Sociological Research for a Dynamic World}.
2021;7:237802312110529.
doi:\href{https://doi.org/10.1177/23780231211052946}{10.1177/23780231211052946}}

\leavevmode\vadjust pre{\hypertarget{ref-mckinnon2021}{}}%
\CSLLeftMargin{18. }%
\CSLRightInline{McKinnon B, Quach C, Dubé Ève, Nguyen CT, Zinszer K.
Social inequalities in {COVID}-19 vaccine acceptance and uptake for
children and adolescents in montreal, canada. \emph{Vaccine}.
2021;39(49):7140-7145.
doi:\href{https://doi.org/10.1016/j.vaccine.2021.10.077}{10.1016/j.vaccine.2021.10.077}}

\leavevmode\vadjust pre{\hypertarget{ref-cnat2022b}{}}%
\CSLLeftMargin{19. }%
\CSLRightInline{Cénat JM, Noorishad PG, Bakombo SM, et al. A systematic
review on vaccine hesitancy in black communities in canada: Critical
issues and research failures. \emph{Vaccines}. 2022;10(11):1937.
doi:\href{https://doi.org/10.3390/vaccines10111937}{10.3390/vaccines10111937}}

\leavevmode\vadjust pre{\hypertarget{ref-sargent2022}{}}%
\CSLLeftMargin{20. }%
\CSLRightInline{Sargent RH, Laurie S, Weakland LF, et al. Use of random
domain intercept technology to track {COVID}-19 vaccination rates in
real time across the united states: Survey study. \emph{Journal of
Medical Internet Research}. 2022;24(7):e37920.
doi:\href{https://doi.org/10.2196/37920}{10.2196/37920}}

\leavevmode\vadjust pre{\hypertarget{ref-deming1940}{}}%
\CSLLeftMargin{21. }%
\CSLRightInline{Deming WE, Stephan FF. On a least squares adjustment of
a sampled frequency table when the expected marginal totals are known.
\emph{The Annals of Mathematical Statistics}. 1940;11(4):427-444.
doi:\href{https://doi.org/10.1214/aoms/1177731829}{10.1214/aoms/1177731829}}

\leavevmode\vadjust pre{\hypertarget{ref-nguyen2022}{}}%
\CSLLeftMargin{22. }%
\CSLRightInline{Nguyen KH, Anneser E, Toppo A, Allen JD, Parott JS,
Corlin L. Disparities in national and state estimates of {COVID}-19
vaccination receipt and intent to vaccinate by race/ethnicity, income,
and age group among adults~\(\geq\)~18~years, united states.
\emph{Vaccine}. 2022;40(1):107-113.
doi:\href{https://doi.org/10.1016/j.vaccine.2021.11.040}{10.1016/j.vaccine.2021.11.040}}

\leavevmode\vadjust pre{\hypertarget{ref-shih2021}{}}%
\CSLLeftMargin{23. }%
\CSLRightInline{Shih SF, Wagner AL, Masters NB, Prosser LA, Lu Y,
Zikmund-Fisher BJ. Vaccine hesitancy and rejection of a vaccine for the
novel coronavirus in the united states. \emph{Frontiers in Immunology}.
2021;12.
doi:\href{https://doi.org/10.3389/fimmu.2021.558270}{10.3389/fimmu.2021.558270}}

\leavevmode\vadjust pre{\hypertarget{ref-cnat2022a}{}}%
\CSLLeftMargin{24. }%
\CSLRightInline{Cénat JM, Noorishad PG, Farahi SMMM, et al. Prevalence
and factors related to {COVID}-19 vaccine hesitancy and unwillingness in
canada: A systematic review and meta-analysis. \emph{Journal of Medical
Virology}. 2022;95(1).
doi:\href{https://doi.org/10.1002/jmv.28156}{10.1002/jmv.28156}}

\leavevmode\vadjust pre{\hypertarget{ref-lumley2011}{}}%
\CSLLeftMargin{25. }%
\CSLRightInline{Lumley T. \emph{Complex Surveys}. John Wiley \& Sons;
2011.}

\leavevmode\vadjust pre{\hypertarget{ref-wickham2019}{}}%
\CSLLeftMargin{26. }%
\CSLRightInline{Wickham H, Averick M, Bryan J, et al. Welcome to the
{tidyverse}. \emph{Journal of Open Source Software}. 2019;4(43):1686.
doi:\href{https://doi.org/10.21105/joss.01686}{10.21105/joss.01686}}

\leavevmode\vadjust pre{\hypertarget{ref-quarto}{}}%
\CSLLeftMargin{27. }%
\CSLRightInline{Allaire J. \emph{Quarto: R Interface to 'Quarto'
Markdown Publishing System}.; 2022.
\url{https://CRAN.R-project.org/package=quarto}}

\leavevmode\vadjust pre{\hypertarget{ref-davis2022}{}}%
\CSLLeftMargin{28. }%
\CSLRightInline{Davis CJ, Golding M, McKay R. Efficacy information
influences intention to take COVID-19 vaccine. \emph{British Journal of
Health Psychology}. 2022;27(2):300-319.
doi:\url{https://doi.org/10.1111/bjhp.12546}}

\leavevmode\vadjust pre{\hypertarget{ref-thelancet2021}{}}%
\CSLLeftMargin{29. }%
\CSLRightInline{Microbe TL. {COVID}-19 vaccines: The pandemic will not
end overnight. \emph{The Lancet Microbe}. 2021;2(1):e1.
doi:\href{https://doi.org/10.1016/s2666-5247(20)30226-3}{10.1016/s2666-5247(20)30226-3}}

\leavevmode\vadjust pre{\hypertarget{ref-kayser2021}{}}%
\CSLLeftMargin{30. }%
\CSLRightInline{Kayser V, Ramzan I. Vaccines and vaccination: History
and emerging issues. \emph{Human Vaccines {\&}amp\(\mathsemicolon\)
Immunotherapeutics}. 2021;17(12):5255-5268.
doi:\href{https://doi.org/10.1080/21645515.2021.1977057}{10.1080/21645515.2021.1977057}}

\leavevmode\vadjust pre{\hypertarget{ref-li2021}{}}%
\CSLLeftMargin{31. }%
\CSLRightInline{Li Q, Wang J, Tang Y, Lu H. Next-generation COVID-19
vaccines: Opportunities for vaccine development and challenges in
tackling COVID-19. \emph{Drug Discoveries \& Therapeutics}.
2021;15(3):118-123.
doi:\href{https://doi.org/10.5582/ddt.2021.0105}{10.5582/ddt.2021.0105}}

\leavevmode\vadjust pre{\hypertarget{ref-tamey2022}{}}%
\CSLLeftMargin{32. }%
\CSLRightInline{Yamey G, Garcia P, Hassan F, et al. It is not too late
to achieve global covid-19 vaccine equity. \emph{{BMJ}}. Published
online March 2022:e070650.
doi:\href{https://doi.org/10.1136/bmj-2022-070650}{10.1136/bmj-2022-070650}}

\leavevmode\vadjust pre{\hypertarget{ref-willis2021}{}}%
\CSLLeftMargin{33. }%
\CSLRightInline{Willis DE, Andersen JA, Bryant-Moore K, et al.
{COVID}-19 vaccine hesitancy: Race/ethnicity, trust, and fear.
\emph{Clinical and Translational Science}. 2021;14(6):2200-2207.
doi:\href{https://doi.org/10.1111/cts.13077}{10.1111/cts.13077}}

\leavevmode\vadjust pre{\hypertarget{ref-skirrow2022}{}}%
\CSLLeftMargin{34. }%
\CSLRightInline{Skirrow H, Barnett S, Bell S, et al. Women's views on
accepting {COVID}-19 vaccination during and after pregnancy, and for
their babies: A multi-methods study in the {UK}. \emph{{BMC} Pregnancy
and Childbirth}. 2022;22(1).
doi:\href{https://doi.org/10.1186/s12884-021-04321-3}{10.1186/s12884-021-04321-3}}

\leavevmode\vadjust pre{\hypertarget{ref-stoler2021}{}}%
\CSLLeftMargin{35. }%
\CSLRightInline{Stoler J, Enders AM, Klofstad CA, Uscinski JE. The
limits of medical trust in mitigating {COVID}-19 vaccine hesitancy among
black americans. \emph{Journal of General Internal Medicine}.
2021;36(11):3629-3631.
doi:\href{https://doi.org/10.1007/s11606-021-06743-3}{10.1007/s11606-021-06743-3}}

\leavevmode\vadjust pre{\hypertarget{ref-khubchandani2021}{}}%
\CSLLeftMargin{36. }%
\CSLRightInline{Khubchandani J, Sharma S, Price JH, Wiblishauser MJ,
Sharma M, Webb FJ. {COVID}-19 vaccination hesitancy in the united
states: A rapid national assessment. \emph{Journal of Community Health}.
2021;46(2):270-277.
doi:\href{https://doi.org/10.1007/s10900-020-00958-x}{10.1007/s10900-020-00958-x}}

\leavevmode\vadjust pre{\hypertarget{ref-bogart2021}{}}%
\CSLLeftMargin{37. }%
\CSLRightInline{Bogart LM, Ojikutu BO, Tyagi K, et al. {COVID}-19
related medical mistrust, health impacts, and potential vaccine
hesitancy among black americans living with {HIV}. \emph{{JAIDS} Journal
of Acquired Immune Deficiency Syndromes}. 2021;86(2):200-207.
doi:\href{https://doi.org/10.1097/qai.0000000000002570}{10.1097/qai.0000000000002570}}

\leavevmode\vadjust pre{\hypertarget{ref-mosby2021}{}}%
\CSLLeftMargin{38. }%
\CSLRightInline{Mosby I, Swidrovich J. Medical experimentation and the
roots of {COVID}-19 vaccine hesitancy among indigenous peoples in
canada. \emph{Canadian Medical Association Journal}.
2021;193(11):E381-E383.
doi:\href{https://doi.org/10.1503/cmaj.210112}{10.1503/cmaj.210112}}

\leavevmode\vadjust pre{\hypertarget{ref-freeman2020}{}}%
\CSLLeftMargin{39. }%
\CSLRightInline{Freeman D, Loe BS, Chadwick A, et al. {COVID}-19 vaccine
hesitancy in the {UK}: The oxford coronavirus explanations, attitudes,
and narratives survey (oceans) {II}. \emph{Psychological Medicine}.
2020;52(14):3127-3141.
doi:\href{https://doi.org/10.1017/s0033291720005188}{10.1017/s0033291720005188}}

\leavevmode\vadjust pre{\hypertarget{ref-pallathadka2022}{}}%
\CSLLeftMargin{40. }%
\CSLRightInline{Pallathadka A, Chang H, Han D. What explains spatial
variations of {COVID}-19 vaccine hesitancy?: A
social-ecological-technological systems approach. \emph{Environmental
Research: Health}. 2022;1(1):011001.
doi:\href{https://doi.org/10.1088/2752-5309/ac8ac2}{10.1088/2752-5309/ac8ac2}}

\leavevmode\vadjust pre{\hypertarget{ref-huang2022}{}}%
\CSLLeftMargin{41. }%
\CSLRightInline{Huang Q, Cutter SL. Spatial-temporal differences of
{COVID}-19 vaccinations in the u.s. \emph{Urban Informatics}. 2022;1(1).
doi:\href{https://doi.org/10.1007/s44212-022-00019-9}{10.1007/s44212-022-00019-9}}

\leavevmode\vadjust pre{\hypertarget{ref-lavoie2022}{}}%
\CSLLeftMargin{42. }%
\CSLRightInline{Lavoie K, Gosselin-Boucher V, Stojanovic J, et al.
Understanding national trends in {COVID}-19 vaccine hesitancy in canada:
Results from five sequential cross-sectional representative surveys
spanning april 2020{\textendash}march 2021. \emph{{BMJ} Open}.
2022;12(4):e059411.
doi:\href{https://doi.org/10.1136/bmjopen-2021-059411}{10.1136/bmjopen-2021-059411}}

\leavevmode\vadjust pre{\hypertarget{ref-dong2022}{}}%
\CSLLeftMargin{43. }%
\CSLRightInline{Dong L, Sahu R, Black R. Governance in the
transformational journey toward integrated healthcare: The case of
ontario. \emph{Journal of Information Technology Teaching Cases}.
Published online December 2022:204388692211473.
doi:\href{https://doi.org/10.1177/20438869221147313}{10.1177/20438869221147313}}

\leavevmode\vadjust pre{\hypertarget{ref-tsasis2012}{}}%
\CSLLeftMargin{44. }%
\CSLRightInline{Tsasis P, Evans JM, Owen S. Reframing the challenges to
integrated care: A complex-adaptive systems perspective.
\emph{International Journal of Integrated Care}. 2012;12(5).
doi:\href{https://doi.org/10.5334/ijic.843}{10.5334/ijic.843}}

\leavevmode\vadjust pre{\hypertarget{ref-dube2022}{}}%
\CSLLeftMargin{45. }%
\CSLRightInline{Dubé E, Gagnon D, MacDonald N. Between persuasion and
compulsion: The case of {COVID}-19 vaccination in canada.
\emph{Vaccine}. 2022;40(29):3923-3926.
doi:\href{https://doi.org/10.1016/j.vaccine.2022.05.053}{10.1016/j.vaccine.2022.05.053}}

\leavevmode\vadjust pre{\hypertarget{ref-macdonald2021}{}}%
\CSLLeftMargin{46. }%
\CSLRightInline{MacDonald NE, Comeau J, Dubé Ève, et al. Royal society
of canada {COVID}-19 report: Enhancing {COVID}-19 vaccine acceptance in
canada. Blais JM, ed. \emph{{FACETS}}. 2021;6:1184-1246.
doi:\href{https://doi.org/10.1139/facets-2021-0037}{10.1139/facets-2021-0037}}

\leavevmode\vadjust pre{\hypertarget{ref-ontario-covid}{}}%
\CSLLeftMargin{47. }%
\CSLRightInline{{Ontario COVID-19 Data Tool}. Accessed February 27,
2023.
\url{https://www.publichealthontario.ca/en/data-and-analysis/infectious-disease/covid-19-data-surveillance/covid-19-data-tool?tab=vaccine}}

\leavevmode\vadjust pre{\hypertarget{ref-carter2022}{}}%
\CSLLeftMargin{48. }%
\CSLRightInline{Carter MA, Biro S, Maier A, Shingler C, Guan TH.
{COVID}-19 vaccine uptake in southeastern ontario, canada: Monitoring
and addressing health inequities. \emph{Journal of Public Health
Management and Practice}. 2022;28(6):615-623.
doi:\href{https://doi.org/10.1097/phh.0000000000001565}{10.1097/phh.0000000000001565}}

\leavevmode\vadjust pre{\hypertarget{ref-hawkins2020}{}}%
\CSLLeftMargin{49. }%
\CSLRightInline{Hawkins D. Differential occupational risk for {COVID}-19
and other infection exposure according to race and ethnicity.
\emph{American Journal of Industrial Medicine}. 2020;63(9):817-820.
doi:\href{https://doi.org/10.1002/ajim.23145}{10.1002/ajim.23145}}

\leavevmode\vadjust pre{\hypertarget{ref-ct2021}{}}%
\CSLLeftMargin{50. }%
\CSLRightInline{Côté D, Durant S, MacEachen E, et al. A rapid scoping
review of {COVID}-19 and vulnerable workers: Intersecting occupational
and public health issues. \emph{American Journal of Industrial
Medicine}. 2021;64(7):551-566.
doi:\href{https://doi.org/10.1002/ajim.23256}{10.1002/ajim.23256}}

\leavevmode\vadjust pre{\hypertarget{ref-mishra2021}{}}%
\CSLLeftMargin{51. }%
\CSLRightInline{Mishra S, Stall NM, Ma H, et al. \emph{A Vaccination
Strategy for Ontario {COVID}-19 Hotspots and Essential Workers}. Ontario
{COVID}-19 Science Advisory Table; 2021.
doi:\href{https://doi.org/10.47326/ocsat.2021.02.26.1.0}{10.47326/ocsat.2021.02.26.1.0}}

\leavevmode\vadjust pre{\hypertarget{ref-ontariohealth}{}}%
\CSLLeftMargin{52. }%
\CSLRightInline{\emph{{A}nnual {B}usiness {P}lan 2022/23}. Ontario
Health;
\url{https://www.ontariohealth.ca/sites/ontariohealth/files/2022-05/OHBusinessPlan22_23.pdf};
2022.}

\leavevmode\vadjust pre{\hypertarget{ref-buajitti2018}{}}%
\CSLLeftMargin{53. }%
\CSLRightInline{Buajitti E, Watson T, Kornas K, Bornbaum C, Henry D,
Rosella LC. Ontario atlas of adult mortality, 1992-2015: Trends in
{L}ocal {H}ealth {I}ntegration {N}etworks. Published online 2018.
\url{https://tspace.library.utoronto.ca/handle/1807/82836}}

\leavevmode\vadjust pre{\hypertarget{ref-anand2022}{}}%
\CSLLeftMargin{54. }%
\CSLRightInline{Anand SS, Arnold C, Bangdiwala SI, et al. Seropositivity
and risk factors for {SARS}-{CoV}-2 infection in a south asian community
in ontario: A cross-sectional analysis of a prospective cohort study.
\emph{{CMAJ} Open}. 2022;10(3):E599-E609.
doi:\href{https://doi.org/10.9778/cmajo.20220031}{10.9778/cmajo.20220031}}

\leavevmode\vadjust pre{\hypertarget{ref-stephenson2022}{}}%
\CSLLeftMargin{55. }%
\CSLRightInline{Stephenson M, Olson SM, Self WH, et al. Ascertainment of
vaccination status by self-report versus source documentation: Impact on
measuring {COVID}-19 vaccine effectiveness. \emph{Influenza and Other
Respiratory Viruses}. 2022;16(6):1101-1111.
doi:\href{https://doi.org/10.1111/irv.13023}{10.1111/irv.13023}}

\end{CSLReferences}



\end{document}
