% Options for packages loaded elsewhere
\PassOptionsToPackage{unicode}{hyperref}
\PassOptionsToPackage{hyphens}{url}
\PassOptionsToPackage{dvipsnames,svgnames,x11names}{xcolor}
%
\documentclass[
  letterpaper,
  DIV=11,
  numbers=noendperiod]{scrartcl}

\usepackage{amsmath,amssymb}
\usepackage{iftex}
\ifPDFTeX
  \usepackage[T1]{fontenc}
  \usepackage[utf8]{inputenc}
  \usepackage{textcomp} % provide euro and other symbols
\else % if luatex or xetex
  \usepackage{unicode-math}
  \defaultfontfeatures{Scale=MatchLowercase}
  \defaultfontfeatures[\rmfamily]{Ligatures=TeX,Scale=1}
\fi
\usepackage{lmodern}
\ifPDFTeX\else  
    % xetex/luatex font selection
\fi
% Use upquote if available, for straight quotes in verbatim environments
\IfFileExists{upquote.sty}{\usepackage{upquote}}{}
\IfFileExists{microtype.sty}{% use microtype if available
  \usepackage[]{microtype}
  \UseMicrotypeSet[protrusion]{basicmath} % disable protrusion for tt fonts
}{}
\makeatletter
\@ifundefined{KOMAClassName}{% if non-KOMA class
  \IfFileExists{parskip.sty}{%
    \usepackage{parskip}
  }{% else
    \setlength{\parindent}{0pt}
    \setlength{\parskip}{6pt plus 2pt minus 1pt}}
}{% if KOMA class
  \KOMAoptions{parskip=half}}
\makeatother
\usepackage{xcolor}
\setlength{\emergencystretch}{3em} % prevent overfull lines
\setcounter{secnumdepth}{-\maxdimen} % remove section numbering
% Make \paragraph and \subparagraph free-standing
\ifx\paragraph\undefined\else
  \let\oldparagraph\paragraph
  \renewcommand{\paragraph}[1]{\oldparagraph{#1}\mbox{}}
\fi
\ifx\subparagraph\undefined\else
  \let\oldsubparagraph\subparagraph
  \renewcommand{\subparagraph}[1]{\oldsubparagraph{#1}\mbox{}}
\fi


\providecommand{\tightlist}{%
  \setlength{\itemsep}{0pt}\setlength{\parskip}{0pt}}\usepackage{longtable,booktabs,array}
\usepackage{calc} % for calculating minipage widths
% Correct order of tables after \paragraph or \subparagraph
\usepackage{etoolbox}
\makeatletter
\patchcmd\longtable{\par}{\if@noskipsec\mbox{}\fi\par}{}{}
\makeatother
% Allow footnotes in longtable head/foot
\IfFileExists{footnotehyper.sty}{\usepackage{footnotehyper}}{\usepackage{footnote}}
\makesavenoteenv{longtable}
\usepackage{graphicx}
\makeatletter
\def\maxwidth{\ifdim\Gin@nat@width>\linewidth\linewidth\else\Gin@nat@width\fi}
\def\maxheight{\ifdim\Gin@nat@height>\textheight\textheight\else\Gin@nat@height\fi}
\makeatother
% Scale images if necessary, so that they will not overflow the page
% margins by default, and it is still possible to overwrite the defaults
% using explicit options in \includegraphics[width, height, ...]{}
\setkeys{Gin}{width=\maxwidth,height=\maxheight,keepaspectratio}
% Set default figure placement to htbp
\makeatletter
\def\fps@figure{htbp}
\makeatother
\newlength{\cslhangindent}
\setlength{\cslhangindent}{1.5em}
\newlength{\csllabelwidth}
\setlength{\csllabelwidth}{3em}
\newlength{\cslentryspacingunit} % times entry-spacing
\setlength{\cslentryspacingunit}{\parskip}
\newenvironment{CSLReferences}[2] % #1 hanging-ident, #2 entry spacing
 {% don't indent paragraphs
  \setlength{\parindent}{0pt}
  % turn on hanging indent if param 1 is 1
  \ifodd #1
  \let\oldpar\par
  \def\par{\hangindent=\cslhangindent\oldpar}
  \fi
  % set entry spacing
  \setlength{\parskip}{#2\cslentryspacingunit}
 }%
 {}
\usepackage{calc}
\newcommand{\CSLBlock}[1]{#1\hfill\break}
\newcommand{\CSLLeftMargin}[1]{\parbox[t]{\csllabelwidth}{#1}}
\newcommand{\CSLRightInline}[1]{\parbox[t]{\linewidth - \csllabelwidth}{#1}\break}
\newcommand{\CSLIndent}[1]{\hspace{\cslhangindent}#1}

\usepackage{booktabs}
\usepackage{longtable}
\usepackage{array}
\usepackage{multirow}
\usepackage{wrapfig}
\usepackage{float}
\usepackage{colortbl}
\usepackage{pdflscape}
\usepackage{tabu}
\usepackage{threeparttable}
\usepackage{threeparttablex}
\usepackage[normalem]{ulem}
\usepackage{makecell}
\usepackage{xcolor}
\usepackage{caption}
%line numbers
%\usepackage{mathpazo}
%\usepackage{rotating}
\usepackage{lineno}
\linenumbers

\usepackage{rotfloat}
\KOMAoption{captions}{tableheading}
\makeatletter
\makeatother
\makeatletter
\makeatother
\makeatletter
\@ifpackageloaded{caption}{}{\usepackage{caption}}
\AtBeginDocument{%
\ifdefined\contentsname
  \renewcommand*\contentsname{Table of contents}
\else
  \newcommand\contentsname{Table of contents}
\fi
\ifdefined\listfigurename
  \renewcommand*\listfigurename{List of Figures}
\else
  \newcommand\listfigurename{List of Figures}
\fi
\ifdefined\listtablename
  \renewcommand*\listtablename{List of Tables}
\else
  \newcommand\listtablename{List of Tables}
\fi
\ifdefined\figurename
  \renewcommand*\figurename{Figure}
\else
  \newcommand\figurename{Figure}
\fi
\ifdefined\tablename
  \renewcommand*\tablename{Table}
\else
  \newcommand\tablename{Table}
\fi
}
\@ifpackageloaded{float}{}{\usepackage{float}}
\floatstyle{ruled}
\@ifundefined{c@chapter}{\newfloat{codelisting}{h}{lop}}{\newfloat{codelisting}{h}{lop}[chapter]}
\floatname{codelisting}{Listing}
\newcommand*\listoflistings{\listof{codelisting}{List of Listings}}
\makeatother
\makeatletter
\@ifpackageloaded{caption}{}{\usepackage{caption}}
\@ifpackageloaded{subcaption}{}{\usepackage{subcaption}}
\makeatother
\makeatletter
\@ifpackageloaded{tcolorbox}{}{\usepackage[skins,breakable]{tcolorbox}}
\makeatother
\makeatletter
\@ifundefined{shadecolor}{\definecolor{shadecolor}{rgb}{.97, .97, .97}}
\makeatother
\makeatletter
\makeatother
\makeatletter
\makeatother
\ifLuaTeX
  \usepackage{selnolig}  % disable illegal ligatures
\fi
\IfFileExists{bookmark.sty}{\usepackage{bookmark}}{\usepackage{hyperref}}
\IfFileExists{xurl.sty}{\usepackage{xurl}}{} % add URL line breaks if available
\urlstyle{same} % disable monospaced font for URLs
\hypersetup{
  pdftitle={Differences in COVID-19 vaccination in the province of Ontario across Health Regions and socio-economic strata},
  pdfauthor={Ariel Mundo Ortiz1,2; Bouchra Nasri1,2,},
  colorlinks=true,
  linkcolor={blue},
  filecolor={Maroon},
  citecolor={Blue},
  urlcolor={Blue},
  pdfcreator={LaTeX via pandoc}}

\title{\textbf{Differences in COVID-19 vaccination in the province of
Ontario across Health Regions and socio-economic strata}}
\author{}
\date{}

\begin{document}
\maketitle
\ifdefined\Shaded\renewenvironment{Shaded}{\begin{tcolorbox}[boxrule=0pt, borderline west={3pt}{0pt}{shadecolor}, breakable, interior hidden, sharp corners, enhanced, frame hidden]}{\end{tcolorbox}}\fi

\textsuperscript{1} Centre de Recherches Mathématiques, University of
Montreal, Montréal, Canada\\
\textsuperscript{2} Department of Social and Preventive Medicine, École
de Santé Publique, University of Montreal, Montréal, Canada

\textsuperscript{*} Correspondence:
\href{mailto:bouchra.nasri@umontreal.ca}{Bouchra Nasri
\textless{}bouchra.nasri@umontreal.ca\textgreater{}}

\hypertarget{abstract}{%
\section{Abstract}\label{abstract}}

The COVID-19 pandemic continues to be a worldwide public health concern.
Although vaccines against this disease were rapidly developed,
vaccination uptake has not ben equal across all the segments of the
population. In particular, it has been shown that there have been
differences in vaccine uptake across different segments of the
population. However, there are also differences in vaccination across
geographical areas, which might be important to consider in the
development of future public health policies against COVID-19. In this
study, we examined the relationship between vaccination status (having
received the first dose of a COVID-19 vaccine), and different
socio-economic and geographical factors. Our results show that during
the last three months of 2021, individuals in certain equity-deserving
groups (visible minorities) were three times less likely to be
vaccinated than White/Caucasian individuals across the province and that
in some cases, within these groups individuals in low income brackets
had significantly higher odds of vaccination when compared to their
peers in high income brackets. Finally, we identified significantly
lower odds of vaccination in the West Health Region of Ontario within
certain equity-deserving groups. This study shows that there is an
ongoing need to better understand and address differences in vaccination
uptake across diverse segments of the population of Ontario that have
been largely impacted by the pandemic.

\hypertarget{keywords}{%
\section*{Keywords}\label{keywords}}
\addcontentsline{toc}{section}{Keywords}

Covid-19, vaccination, survey, socio-economic factors, visible
minorities.

\hypertarget{background}{%
\section{Background}\label{background}}

As of May of 2023 there have been 765 million confirmed cases of
COVID-19 around the world, including 6.8 million
deaths\textsuperscript{\protect\hyperlink{ref-WHO-Covid}{1}}. Although
this disease is no longer categorized as a global health emergency by
the World Health Organization
(WHO)\textsuperscript{\protect\hyperlink{ref-rigby2023}{2}}, there is
ongoing concern due to continued transmission, surges in cases and
deaths due to new
variants\textsuperscript{\protect\hyperlink{ref-un2023}{3}}, and
weaknesses in health systems around the world that could be exploited by
a novel virus or another public health emergency in the
future\textsuperscript{\protect\hyperlink{ref-mackey2021}{4}}.

In particular, a major weakness that has received attention during the
pandemic has been related to inequalities in vaccine uptake. The rapid
development of vaccines against COVID-19 initially brought the hope of a
rapid end to the pandemic due to the start of vaccination campaigns in
certain parts of the world toward the end of
2020\textsuperscript{\protect\hyperlink{ref-thelancet2021}{5}--\protect\hyperlink{ref-tanne2020}{8}})
but inequalities in vaccine uptake made these pharmaceutical
interventions ultimately unable to replicate the experience of smallpox,
where vaccination on a global scale and was crucial to control this
disease\textsuperscript{\protect\hyperlink{ref-kayser2021}{9}}.

This problematic is a multifaceted issue resulting from a combination of
factors, among which are failed public health
measures\textsuperscript{\protect\hyperlink{ref-li2021}{10}}, inequality
in vaccine access between high- and low-income
countries\textsuperscript{\protect\hyperlink{ref-gerretsen2021}{11},\protect\hyperlink{ref-tamey2022}{12}},
and vaccine
hesitancy\textsuperscript{\protect\hyperlink{ref-nafilyan2021}{13}}.
Furthermore, it is well established that this issue has affected in
particular individuals in certain equity-deserving groups (e.g., Black,
Asian, or Indigenous) as well as individuals with socio-economic
disadvantages\textsuperscript{\protect\hyperlink{ref-willis2021}{14}--\protect\hyperlink{ref-hussain2022}{20}}.

Reasons given for this inequality have included medical mistrust due to
systemic medical
racism\textsuperscript{\protect\hyperlink{ref-stoler2021}{16},\protect\hyperlink{ref-mosby2021}{21}},
mistrust in
vaccines\textsuperscript{\protect\hyperlink{ref-willis2021}{14}}, and
the influence of conspiracy
theories\textsuperscript{\protect\hyperlink{ref-mosby2021}{21}--\protect\hyperlink{ref-freeman2020}{23}}.
However, it is important to also consider that vaccination uptake can be
influenced by geographical (spatial) factors. In this regard,
differences in COVID-19 vaccination rates have been associated with
varied regional attitudes towards
vaccination\textsuperscript{\protect\hyperlink{ref-malik2020}{24}},
spatial differences in vaccine access and supply, vaccination location
availability, and lack of prioritization of areas where vulnerable
groups
reside\textsuperscript{\protect\hyperlink{ref-bogoch2022}{7},\protect\hyperlink{ref-nguyen2021}{25}}.
Other studies have also shown heterogeneity in vaccine uptake within
small governmental administrative units such as
counties\textsuperscript{\protect\hyperlink{ref-mollalo2021}{26}--\protect\hyperlink{ref-bhuiyan2022}{29}},
and that accounting for geographical differences in vaccination can help
predict patterns of booster
uptake\textsuperscript{\protect\hyperlink{ref-wood2022}{30}}.

However, such analyses have been carried mostly in territories outside
of Canada, where available studies have been focused in certain cities
(such as Toronto\textsuperscript{\protect\hyperlink{ref-choi2021}{31}},
or Montreal\textsuperscript{\protect\hyperlink{ref-mckinnon2021}{32}}),
or have explored differences at a province-wide
level\textsuperscript{\protect\hyperlink{ref-guay2022}{18}}. Therefore,
there is a need for studies that explore spatial differences in
vaccination within the Canadian territory and that consequently, can
help identify disparities that need to be addressed within specific
areas in each province.

This need is specially important in the case of Ontario, the most
populated province of Canada. Between 2007 and 2019, Ontario managed
healthcare access to its inhabitants using 14 intra-provincial divisions
called the Local Health Integration Networks (LHINs), which aimed to
provide an integrated health system for the province. However, this
approach was complex and bureaucratic, and resulted in excessive
expenditures, disparities in mortality rates, the deterioration of
certain performance indicators such as wait times and hospital
readmissions, fragmented electronic health systems, the decline of
performance indicators, and inequities in health services
access\textsuperscript{\protect\hyperlink{ref-tsasis2012}{33}--\protect\hyperlink{ref-lysyk2016}{37}}.
Therefore, with the intent of better organizing and delivering care in
late 2019 the provincial government eliminated the LHINs and
incorporated the areas covered by them into six larger Health Regions
(North East, North West, Central, Toronto, West, and
East)\textsuperscript{\protect\hyperlink{ref-dong2022}{35}}.

Because the relatively recent adoption of the Health Region model and
its alignment with the onset of the COVID-19 pandemic, there is a need
to analyze if there are ongoing disparities in health access under this
approach that need to be addressed before they are exploited by a new
disease or public health threat. In this regard, previous research has
highlighted disparities in the level of activity of each Health
Region\textsuperscript{\protect\hyperlink{ref-sethuram2023}{38}}.
Therefore, analyzing differences in vaccination uptake within the Health
Regions and can help identify which socio-demographic groups are the
most vulnerable and what areas of the province deserve special attention
by decision-makers.

Therefore, in this study we hypothesized that there were differences in
vaccination uptake between the different Health Regions of Ontario
between October of 2021 and January of 2022. By including socio-economic
factors in our analysis, we aimed at identifying in which groups these
differences were significant in order to provide an assessment of the
current state of healthcare access in Ontario.

\hypertarget{methods}{%
\section{Methods}\label{methods}}

\hypertarget{sec-data}{%
\subsection{Data and Methods}\label{sec-data}}

We used data from the \emph{Survey of COVID-19 related Behaviours and
Attitudes}, a repeated cross sectional survey focused on the Canadian
province of Ontario that was commissioned by the Fields Institute for
Research in Mathematical Sciences and the Mathematical Modelling of
COVID-19 Task Force under ethical guidance from the University of
Toronto, and which ran between September 30th, 2021 and January
17th,2022. The survey collected socio-economic information from
participants (Table~\ref{tbl-descriptive-stats}), their location
(nearest municipality, as shown in Figure~\ref{fig-map}), the date of
access to the survey, and asked information on vaccination status by
using the question ``Have you received the first dose of the COVID
vaccine?'', with possible answers ``yes'' and ``no''. The original
dataset contained 39,029 observations.

By design, the survey allowed respondents to exit at any time and
deployed the questions randomly, which resulted in \(\approx\) 84\% of
the observations having multiple missing answers or being incomplete.
Therefore, we selected 6,343 observations that were labeled as
``complete'' in the dataset and that had answers for all covariates
considered in our analysis. Later, we matched the city of each
observation with its corresponding LHIN and Health Region, and removed
observations from areas with low representation (254 observations
corresponding to the North West and North East Health Regions). Finally,
we removed outliers from the data (19 observations of individuals with
household size of 1 and income above 110,000 CAD in the original
dataset). After all the preliminary analyses indicated above, the total
number of observations used for analysis was 6,236 and included the
East, Central, Toronto, and West Health Regions covering the period
between October 1st,2021 and January 17, 2022. The original dataset,
clean dataset, and details on the data cleaning process are described in
detail in the \href{https://github.com/aimundo/Fields_COVID-19/}{GitHub
repository} for this paper.

\begin{figure}

{\centering \includegraphics[width=\textwidth,height=1\textheight]{../data/map_data/map_cairo.pdf}

}

\caption{\label{fig-map}Geographic representation of the data collected
by the \emph{Survey of COVID-19 related Behaviours and Attitudes},
collected by the Fields Institute in Ontario. The municipalities from
where survey participants provided answers appear as points. The Health
six Regions are color-coded. Internal boundaries within certain Health
Regions indicate areas that belonged to the Local Integrated Health
Networks (LHINs), the geographic areas for healthcare in Ontario before
the adoption of the Health Regions.}

\end{figure}

\hypertarget{statistical-analyses}{%
\subsection{Statistical analyses}\label{statistical-analyses}}

We used a logistic regression model to examine the impact of the Health
Regions in vaccination rates while considering the socio-economic
factors and and months covered by the survey
(Table~\ref{tbl-descriptive-stats}) and certain interactions (Race and
Health Region and Race and income), as previous studies have shown that
socio-economic factors and their interactions are significant predictors
of intent of vaccination and vaccination
status\textsuperscript{\protect\hyperlink{ref-nguyen2022}{39}--\protect\hyperlink{ref-cnat2022a}{41}}.
Because we identified differences in representativity between the survey
data and the estimates from the Census, we used an iterative
proportional fitting procedure
(\emph{raking})\textsuperscript{\protect\hyperlink{ref-deming1940}{42}}
to correct the data using data from the Census and Health Region
population totals; and fitted the regression model to the uncorrected
and corrected data. Details regarding the correction can be found in the
Appendix. All analyses were conducted in R 4.2.2 using the packages
\texttt{survey}\textsuperscript{\protect\hyperlink{ref-lumley2011}{43}},\texttt{tidyverse}\textsuperscript{\protect\hyperlink{ref-wickham2019}{44}},
\texttt{quarto}\textsuperscript{\protect\hyperlink{ref-quarto}{45}},
\texttt{modelsummary}\textsuperscript{\protect\hyperlink{ref-modelsummary}{46}},
and
\texttt{gtsummary}\textsuperscript{\protect\hyperlink{ref-gtsummary}{47}}.

\hypertarget{results}{%
\section{Results}\label{results}}

\hypertarget{sample-characteristics}{%
\subsection{Sample Characteristics}\label{sample-characteristics}}

Table~\ref{tbl-descriptive-stats} shows the characteristics of the data
from the Fields COVID-19 survey used for analysis. The sample contained
6,236 observations, from which 24.8\% (1,547) corresponded to
individuals that reported not having received the first dose of the
vaccine. Vaccination rates ranged between 71-79\% across all household
income brackets, age groups, Health Regions, and the months considered
in the survey. However, the highest vaccination rates in each category
were reported by individuals in the highest income bracket (79\%), those
between 16 and 34 years of age (77\%), individuals that lived in the
East Health Region (77\%), and during January of 2022 (78\%). Between
racial/ethnic groups, the highest vaccination rate was reported by
White/Caucasian individuals (84\%), against vaccination rates between
63-66\% reported in the case of Arab/Middle Eastern, Black, Indigenous,
Latin American individuals, and those that reported belonging to
``Other'' racial groups, which included Southeast Asian, Filipino, West
Asian, and minorities not identified elsewhere.

\hypertarget{tbl-descriptive-stats}{}
\setlength{\LTpost}{0mm}
\begin{longtable}{lccc}
\caption{\label{tbl-descriptive-stats}Descriptive Statistics of the Fields COVID-19 Survey (by Vaccination
Status) }\tabularnewline

\toprule
\textbf{Variable} & \textbf{no}, N = 1,547\textsuperscript{\textit{1}} & \textbf{yes}, N = 4,689\textsuperscript{\textit{1}} & \textbf{p-value}\textsuperscript{\textit{2}} \\ 
\midrule
Income (CAD) &  &  & <0.001 \\ 
    60000 and above & 542 (21\%) & 1,996 (79\%) &  \\ 
    25000-59999 & 347 (25\%) & 1,046 (75\%) &  \\ 
    under 25000 & 658 (29\%) & 1,647 (71\%) &  \\ 
Age Group &  &  & 0.002 \\ 
    16-34 & 645 (23\%) & 2,117 (77\%) &  \\ 
    35-54 & 411 (24\%) & 1,305 (76\%) &  \\ 
    55 and over & 491 (28\%) & 1,267 (72\%) &  \\ 
Health Region &  &  & 0.3 \\ 
    Toronto & 593 (26\%) & 1,709 (74\%) &  \\ 
    Central & 372 (26\%) & 1,083 (74\%) &  \\ 
    East & 236 (23\%) & 783 (77\%) &  \\ 
    West & 346 (24\%) & 1,114 (76\%) &  \\ 
Month &  &  & <0.001 \\ 
    October & 469 (27\%) & 1,263 (73\%) &  \\ 
    November & 376 (28\%) & 980 (72\%) &  \\ 
    December & 181 (24\%) & 565 (76\%) &  \\ 
    January & 521 (22\%) & 1,881 (78\%) &  \\ 
Race &  &  & <0.001 \\ 
    White/Caucasian & 354 (16\%) & 1,871 (84\%) &  \\ 
    Arab/Middle Eastern & 111 (34\%) & 220 (66\%) &  \\ 
    Black & 159 (34\%) & 303 (66\%) &  \\ 
    East Asian/Pacific Islander & 94 (19\%) & 404 (81\%) &  \\ 
    Indigenous & 112 (37\%) & 194 (63\%) &  \\ 
    Latin American & 99 (34\%) & 195 (66\%) &  \\ 
    Mixed & 177 (30\%) & 411 (70\%) &  \\ 
    Other\textsuperscript{\textit{3}} & 315 (34\%) & 606 (66\%) &  \\ 
    South Asian & 126 (21\%) & 485 (79\%) &  \\ 
\bottomrule
\end{longtable}
\begin{minipage}{\linewidth}
\textsuperscript{\textit{1}}n (\%)\\
\textsuperscript{\textit{2}}Pearson's Chi-squared test\\
\textsuperscript{\textit{3}}Southeast Asian, Filipino, West Asian,
and minorities not identified elsewhere according to the Census.\\
\end{minipage}

\hypertarget{multivariate-regression}{%
\subsection{Multivariate Regression}\label{multivariate-regression}}

Figure~\ref{fig-models} presents the estimates (as odd ratios) from the
logistic regression models for vaccination status using the
socio-demographic factors collected by the survey, and their
interactions. Generally speaking, lower odds of vaccination were
identified in both cases in individuals characterized by a low household
income, or that identified as part of equity-deserving groups. However,
the magnitude of the estimates differed between the uncorrected and
corrected models and more importantly, there were differences in the
statistical significance of certain estimates before and after the
correction. Specifically, the uncorrected model showed significant
differences in vaccination odds between the age groups considered, the
East Health Region, Latin American individuals with a household income
under CAD 25,000, and Indigenous individuals living in the Central
Health Region (Figure~\ref{fig-models},B) but these were deemed non
statistically-significant after the correction.

\begin{figure}

\includegraphics{main_files/figure-pdf/fig-models-1.pdf} \hfill{}

\caption{\label{fig-models}Coefficient estimates and confidence
intervals for the uncorrected model. Only statistically significant
interaction terms are shown. Full interaction terms can be found in
Supplementary Figures A-3 and A-4.}

\end{figure}

However, significantly lower odds of vaccination were identified in the
corrected model for those with a household income under CAD 25,000
(OR=0.37, CI={[}0.27,0.51{]}) and those with an income between CAD
25,000 and 59,999 (OR=0.58, CI={[}0.42,0.81{]}). Additionally,
individuals who identified as Arab/Middle Eastern, Black, Latin
American, of mixed background, or that belonged to other racial groups
(a category that included Southeast Asian, Filipino, West Asian, and
minorities not identified elsewhere), had significantly lower odds of
vaccination than those in the White/Caucasian group (ORs and CIs=0.28
{[}0.16,0.51{]}, 0.27 {[}0.16,0.45{]}, 0.40 {[}0.21,0.76{]}, 0.53
{[}0.30,0.92{]}, 0.23 {[}0.15,0.36{]}). Additionally, individuals that
reported living in the Central and West Health Regions had higher odds
of vaccination than those in the Health Region of Toronto (ORs and
CIs=1.61 {[}1.10,2.34{]}, and 1.59 {[}1.16,2.19{]}, respectively).

Interestingly, individuals in equity-deserving groups with a household
income below CAD 25,000 had higher odds of vaccination (when compared to
those with a household income above CAD 60,000). This held true in the
case of Arab/Middle Eastern individuals (OR=34, CI={[}1.70,6.79{]}),
Black individuals (OR=3.81, CI={[}2.05, 7.09{]}), and those in other
racial or ethnic groups (OR=3.19, CI={[}2.00,5.09{]}). Additionally,
individuals with an income between CAD 25,000 and 59,999 in the
Arab/Middle Eastern and other racial ethnic groups also had higher odds
of vaccination than their high-income peers (ORs and CIs=6.96
{[}2.67,18.16{]}, and 3.5 {[}1.85,6.62{]}).

Finally, significantly lower odds of vaccination were identified (when
compared to the Toronto Health Region) for Black individuals in the
Central Health Region (OR=0.39, CI={[}0.2,0.75{]}), Arab/Middle Eastern
individuals in the East Health Region (OR=0.41 {[}0.17, 0.98{]}), and in
the Indigenous and mixed groups in the West Health Region (ORs and
CIs={[}0.31 {[}0.14, 0.7{]} and 0.38 {[}0.19, 0.76{]}, respectively).

\hypertarget{discussion}{%
\section{Discussion}\label{discussion}}

In this study we hypothesized that differences in COVID-19 vaccination
uptake were present between the Health Regions during between late 2021
and early 2022. Our goal was to determine which socio-demographic groups
could be impacted by these disparities in order to provide
decision-makers with information that could be used to develop policies
focused on reducing or eliminating these differences and ensure that the
Health Region model is able to fulfill its mission of improving health
access for all Ontarians.

Our results show that indeed, there were differences in vaccination odds
across Ontario in certain socio-demographic groups. Specifically, those
who identified as Arab/Middle Eastern, Black, Latin American, having
mixed racial or ethnic background, or that belonged to other groups not
explicitly included in the survey (Southeast Asian, Filipino, West
Asian, and minority groups not identified elsewhere) had vaccination
odds that were between a third and a half of that of individuals that
identified as White or Caucasian (Figure~\ref{fig-models}). These
results are consistent with previous studies that have shown lower
vaccination rates in individuals with the same socio-demographic
characteristics\textsuperscript{\protect\hyperlink{ref-guay2022}{18}--\protect\hyperlink{ref-hussain2022}{20},\protect\hyperlink{ref-carter2022}{48}}.

Lower vaccine uptake in the socio-demographic groups indicated above may
be influenced in part, by vaccine hesitancy and refusal, which have been
associated in equity-deserving Canadian individuals to concerns on
vaccine safety, effectiveness, and experiences of racial discrimination
in health
settings\textsuperscript{\protect\hyperlink{ref-cnat2022a}{41},\protect\hyperlink{ref-basta2022}{49}--\protect\hyperlink{ref-cnat2023}{51}}.
However, it has been shown that structural barriers also play an
important role in vaccination uptake. In the case of equity-deserving
individuals, such barriers include complex scheduling systems, language
barriers, lack of adequate public transportation, and lack of accessible
vaccination
sites\textsuperscript{\protect\hyperlink{ref-njoku2021}{52}}. In this
regard, it is interesting to note that vaccination venues were scarce in
low socio-economic areas that had the highest burden of COVID-19 in
Toronto and other regions of Ontario around the time covered by the
survey\textsuperscript{\protect\hyperlink{ref-bogoch2022}{7},\protect\hyperlink{ref-iveniuk2021}{53}},
and that pharmacies in the Peel region (an area identified as a
``hotspot'' with high numbers of essential workers and multigenerational
households) could not keep up with
demand\textsuperscript{\protect\hyperlink{ref-gill2022}{54}}. This
suggests that the observed differences are associated to disparities in
vaccine access that were present during the period covered by the
survey.

Interestingly, whereas overall self-reported vaccination rates were
found to be statistically significantly lower in various racial minority
groups when compared to White/Caucasian individuals, the change in odds
of vaccination within certain racial groups and income strata was
actually positive, in contrast to the White/Caucasian group, where
vaccination odds decreased in income brackets below CAD 60,000
(Supplementary Figure A-5). Specifically, individuals in low income
brackets that belonged to Arab/Middle Eastern, Black, or other minority
groups had higher odds of vaccination that their peers with an income
above 60,000 CAD.

This result is likely reflects in part the fact that individuals in
racial minority groups tend to perform occupations that have been deemed
as ``essential'' in the context of the
pandemic\textsuperscript{\protect\hyperlink{ref-hawkins2020}{55},\protect\hyperlink{ref-ct2021}{56}},
which include workers in the areas of grocery stores, gas stations,
warehouses, distribution, and manufacturing, all being occupations for
which an income within the significant brackets identified in the
analysis is to be expected. In Ontario, these workers had priority for
COVID-19
vaccination\textsuperscript{\protect\hyperlink{ref-mishra2021}{57}}; and
there is evidence of interventions by vaccination staff in certain parts
of the province to encourage vaccination uptake by these
individuals\textsuperscript{\protect\hyperlink{ref-gill2022}{54}}. These
facts, combined with evidence of increased trends in vaccination in this
group
elsewhere\textsuperscript{\protect\hyperlink{ref-nguyen2021b}{58}},
suggest that the type of occupation in individuals of equity-deserving
groups played an important role in increasing the odds of vaccination.

However, the results also indicate that the place of habitation affected
the odds of vaccination for certain equity-deserving groups (interaction
term of Health Region and Race, Figure~\ref{fig-models},B).
Specifically, this held true in the case of individuals identifying as
Indigenous or with mixed racial background in the West Health Region,
Black individuals in the Central Health Region, and Arab/Middle Eastern
individuals in the East Health Region Figure~\ref{fig-models}. For these
individuals, vaccination odds were lower when compared to the Toronto
Health Region (Supplementary Figure A-6). We indicate next some
contributing factors that might help provide context to these results.

First, in this case it is useful to analyze the data considering the
LHINs in each Health Region, because most studies in the literature
focused on Ontario use the LHINs as the base of their analyses. The West
Health Region covers the area previously occupied by the Hamilton
Niagara Haldimand Brant, South West, and Waterloo Wellington LHINs,
whereas the East Health Region covers the area of the former Champlain
and Central East LHINs. Previous research has identified health
disparities in these (mostly rural) regions, such as unequal
distribution of primary care providers, increased mortality, and low
pharmacist
availability\textsuperscript{\protect\hyperlink{ref-shah2019}{59}--\protect\hyperlink{ref-timony2022}{61}}.

Furthermore, there is an ongoing challenge for the health system of the
province with regard to personalized healthcare for marginalized
individuals. For example, the West Health Region has only two Aboriginal
Health Access Centres (community-led primary healthcare organizations
focused on First Nations, Métis, and Inuit communities) to provide care
to an estimated 100,000 Indigenous individuals living in the
area\textsuperscript{\protect\hyperlink{ref-ontariohealth}{62}}. Lack of
access to personalized healthcare affects individuals that may mistrust
the traditional healthcare system due to systemic racism or oppression,
which is known to be the case for Indigenous and Black individuals in
Canada, as these rationales have been associated to observed lower
vaccination rates among these
groups\textsuperscript{\protect\hyperlink{ref-smylie2022}{63},\protect\hyperlink{ref-eissa2021}{64}}.
Taken together, this suggests that healthcare disparities specific to
these equity-deserving groups in certain parts of the province impacted
vaccination uptake, and highlights the need of investments in the Health
Regions focused on resources, infrastructure, and specially personnel
that can deliver personalized care to marginalized communities, as it
has been shown that such efforts have improved trust in vaccination in
visible
minorities\textsuperscript{\protect\hyperlink{ref-schafferderoo2020}{65}}.

There are some limitations to the present study. First, the data
collection design, which allowed respondents to withdraw from the survey
at any point, and that deployed the questions in a random manner
resulted in an elevated number of missing observations without a
definite pattern and complicated the implementation of sensitivity
analyses. Therefore, we focused on entries that had complete answers,
and corrected the data using population-wide information from the
Census. More granular corrections would be needed to obtain more
accurate estimates. For example, our analysis identified higher odds of
vaccination in the Central and West Health Regions, but in this case
these differences are likely to be driven by the proportion of
White/Caucasian individuals, who had higher vaccination rates that other
racial groups. Correcting for each racial/ethnic group in each Health
Region can provide a more accurate estimation of region-wide vaccination
rates but unfortunately at the moment this correction cannot be
implemented as such stratification is has not been implemented in the
Census.

Additionally, our analysis did not consider the North West and North
East Health Regions, due to the low number of entries from these areas
in the survey (Figure~\ref{fig-map}). Low representation is expected as
these regions as they only account for 5\% of the total population of
Ontario, but in contrast, they have the highest proportion of Indigenous
inhabitants\textsuperscript{\protect\hyperlink{ref-ontariohealth}{62}}.
In the context of personalized care, there is a need for collecting data
that focuses on these Health Regions where additional health disparities
might be present and possibly understudied.

The results in this study are based on self-reported data, where bias
might be present. However, because in the context of COVID-19 it has
been shown that good agreement exists between self-reported and
documented vaccination
status\textsuperscript{\protect\hyperlink{ref-stephenson2022}{66}}, we
believe that our data was able to provide a valid sample of vaccination
uptake in the province. This is supported by the
statistically-significant higher vaccination odds identified for January
of 2022 in the model, which are consistent with province-wide trends
reported by Public Health Ontario (which show a 4\% increase between
early December and January, in contrast to a 2.5\% increase between
October and
November\textsuperscript{\protect\hyperlink{ref-ontario-covid}{67}});
however, the short time window constitutes essentially a ``snapshot''
view of the evolution of the disease, and additional data would be
needed in order to obtain estimates per racial/ethnic group over time
across all Health Regions that can help inform the existence of other
health disparities.

Nonetheless, the results presented here can serve as a starting point to
motivate the collection of robust longitudinal data that can be used to
quantify geographical and temporal differences within vulnerable
segments of the population, and that can be used to inform the
development of adequate public health policies within the province of
Ontario or across other provinces in Canada that aim to minimize
disparities in health access.

\hypertarget{conclusion}{%
\section{Conclusion}\label{conclusion}}

The implementation of the Health Regions in Ontario aimed at reducing
the bureaucratic complexity and health disparities identified under the
LHIN model. However, there are currently multiple challenges that need
to be addressed to ensure that the new model is able to improve
healthcare for the inhabitants of the province. First, the fact that
each Health Region now covers a large geographical area that was served
by multiple LHINs in the past creates a complex socio-demographic
landscape that is different in each case due the different levels of
rurality and representation of equity-deserving groups that are now
within each Health Region. So far, the evidence collected during the
COVID-19 pandemic indicates that differences in vaccination uptake are
associated to a lack of infrastructure and resources that can adequately
support personalized care to marginalized individuals. In the near
future, health decision-makers will need to consider the implementation
of policies that are focused on addressing this problematic.

Moreover, the recent nature in the adaption of the Health Region poses a
challenge for researchers in the acquisition of data and information
that can be used to analyze the performance of the new system. From one
side, the Health Regions have not been incorporated as part of Census
data (LHINs were considered before in the Census), and this impact the
amount and level of detail of available information. Currently, the only
demographic information available for each Health Region is provided by
Ontario Health (the agency that administers the Health Regions) but this
information only provides general estimates that do not allow for
detailed analyses on performance indicators (such as hospitalizations,
readmissions, and trends in chronic disease incidence) between the
regions. Without open information, it is impossible to assess the level
of success of the Health Region model, which is critical considering
that such evaluations have not been part of the Annual Reports of the
Auditor General of Ontario, which in the past analyzed the performance
of the LHINs and pointed to ongoing needs and failures in the system.

The Health Region model will only by successful if it ensures that
healthcare improves across all segments of the population of Ontario,
particularly in the event of a future public health emergency or
pandemic where so far, based on the experience of the COVID-19 pandemic,
equity-deserving individuals have been disproportionately affected.

\hypertarget{references}{%
\section{References}\label{references}}

\hypertarget{refs}{}
\begin{CSLReferences}{0}{0}
\leavevmode\vadjust pre{\hypertarget{ref-WHO-Covid}{}}%
\CSLLeftMargin{1. }%
\CSLRightInline{{World Health Organization Coronavirus (COVID-19)
Dashboard}. Accessed May 11, 2023. \url{https://covid19.who.int/}}

\leavevmode\vadjust pre{\hypertarget{ref-rigby2023}{}}%
\CSLLeftMargin{2. }%
\CSLRightInline{Rigby J, Satija B. {WHO} declares end to COVID global
health emergency. \emph{Reuters}. Published online May 8, 2023. Accessed
May 11, 2022.
\url{https://www.reuters.com/business/healthcare-pharmaceuticals/covid-is-no-longer-global-health-emergency-who-2023-05-05/}}

\leavevmode\vadjust pre{\hypertarget{ref-un2023}{}}%
\CSLLeftMargin{3. }%
\CSLRightInline{Nations U. {WHO} chief declares end to COVID-19 as a
global health emergency. \emph{UN News}. Published online May 5, 2023.
Accessed May 11, 2022.
\url{https://news.un.org/en/story/2023/05/1136367}}

\leavevmode\vadjust pre{\hypertarget{ref-mackey2021}{}}%
\CSLLeftMargin{4. }%
\CSLRightInline{Mackey K, Ayers CK, Kondo KK, et al. Racial and ethnic
disparities in {COVID}-19{\textendash}related infections,
hospitalizations, and deaths. \emph{Annals of Internal Medicine}.
2021;174(3):362-373.
doi:\href{https://doi.org/10.7326/m20-6306}{10.7326/m20-6306}}

\leavevmode\vadjust pre{\hypertarget{ref-thelancet2021}{}}%
\CSLLeftMargin{5. }%
\CSLRightInline{Microbe TL. {COVID}-19 vaccines: The pandemic will not
end overnight. \emph{The Lancet Microbe}. 2021;2(1):e1.
doi:\href{https://doi.org/10.1016/s2666-5247(20)30226-3}{10.1016/s2666-5247(20)30226-3}}

\leavevmode\vadjust pre{\hypertarget{ref-davis2022}{}}%
\CSLLeftMargin{6. }%
\CSLRightInline{Davis CJ, Golding M, McKay R. Efficacy information
influences intention to take COVID-19 vaccine. \emph{British Journal of
Health Psychology}. 2022;27(2):300-319.
doi:\url{https://doi.org/10.1111/bjhp.12546}}

\leavevmode\vadjust pre{\hypertarget{ref-bogoch2022}{}}%
\CSLLeftMargin{7. }%
\CSLRightInline{Bogoch II, Halani S. {COVID}-19 vaccines: A geographic,
social and policy view of vaccination efforts in ontario, canada.
\emph{Cambridge Journal of Regions, Economy and Society}. Published
online November 2022.
doi:\href{https://doi.org/10.1093/cjres/rsac043}{10.1093/cjres/rsac043}}

\leavevmode\vadjust pre{\hypertarget{ref-tanne2020}{}}%
\CSLLeftMargin{8. }%
\CSLRightInline{Tanne JH. Covid-19: {FDA} panel votes to authorise
pfizer {BioNTech} vaccine. \emph{{BMJ}}. Published online December
2020:m4799.
doi:\href{https://doi.org/10.1136/bmj.m4799}{10.1136/bmj.m4799}}

\leavevmode\vadjust pre{\hypertarget{ref-kayser2021}{}}%
\CSLLeftMargin{9. }%
\CSLRightInline{Kayser V, Ramzan I. Vaccines and vaccination: History
and emerging issues. \emph{Human Vaccines {\&}amp\(\mathsemicolon\)
Immunotherapeutics}. 2021;17(12):5255-5268.
doi:\href{https://doi.org/10.1080/21645515.2021.1977057}{10.1080/21645515.2021.1977057}}

\leavevmode\vadjust pre{\hypertarget{ref-li2021}{}}%
\CSLLeftMargin{10. }%
\CSLRightInline{Li Q, Wang J, Tang Y, Lu H. Next-generation COVID-19
vaccines: Opportunities for vaccine development and challenges in
tackling COVID-19. \emph{Drug Discoveries \& Therapeutics}.
2021;15(3):118-123.
doi:\href{https://doi.org/10.5582/ddt.2021.0105}{10.5582/ddt.2021.0105}}

\leavevmode\vadjust pre{\hypertarget{ref-gerretsen2021}{}}%
\CSLLeftMargin{11. }%
\CSLRightInline{Gerretsen P, Kim J, Caravaggio F, et al. Individual
determinants of {COVID}-19 vaccine hesitancy. Inbaraj LR, ed.
\emph{{PLOS} {ONE}}. 2021;16(11):e0258462.
doi:\href{https://doi.org/10.1371/journal.pone.0258462}{10.1371/journal.pone.0258462}}

\leavevmode\vadjust pre{\hypertarget{ref-tamey2022}{}}%
\CSLLeftMargin{12. }%
\CSLRightInline{Yamey G, Garcia P, Hassan F, et al. It is not too late
to achieve global covid-19 vaccine equity. \emph{{BMJ}}. Published
online March 2022:e070650.
doi:\href{https://doi.org/10.1136/bmj-2022-070650}{10.1136/bmj-2022-070650}}

\leavevmode\vadjust pre{\hypertarget{ref-nafilyan2021}{}}%
\CSLLeftMargin{13. }%
\CSLRightInline{Nafilyan V, Dolby T, Razieh C, et al. Sociodemographic
inequality in {COVID}-19 vaccination coverage among elderly adults in
england: A national linked data study. \emph{{BMJ} Open}.
2021;11(7):e053402.
doi:\href{https://doi.org/10.1136/bmjopen-2021-053402}{10.1136/bmjopen-2021-053402}}

\leavevmode\vadjust pre{\hypertarget{ref-willis2021}{}}%
\CSLLeftMargin{14. }%
\CSLRightInline{Willis DE, Andersen JA, Bryant-Moore K, et al.
{COVID}-19 vaccine hesitancy: Race/ethnicity, trust, and fear.
\emph{Clinical and Translational Science}. 2021;14(6):2200-2207.
doi:\href{https://doi.org/10.1111/cts.13077}{10.1111/cts.13077}}

\leavevmode\vadjust pre{\hypertarget{ref-skirrow2022}{}}%
\CSLLeftMargin{15. }%
\CSLRightInline{Skirrow H, Barnett S, Bell S, et al. Women's views on
accepting {COVID}-19 vaccination during and after pregnancy, and for
their babies: A multi-methods study in the {UK}. \emph{{BMC} Pregnancy
and Childbirth}. 2022;22(1).
doi:\href{https://doi.org/10.1186/s12884-021-04321-3}{10.1186/s12884-021-04321-3}}

\leavevmode\vadjust pre{\hypertarget{ref-stoler2021}{}}%
\CSLLeftMargin{16. }%
\CSLRightInline{Stoler J, Enders AM, Klofstad CA, Uscinski JE. The
limits of medical trust in mitigating {COVID}-19 vaccine hesitancy among
black americans. \emph{Journal of General Internal Medicine}.
2021;36(11):3629-3631.
doi:\href{https://doi.org/10.1007/s11606-021-06743-3}{10.1007/s11606-021-06743-3}}

\leavevmode\vadjust pre{\hypertarget{ref-khubchandani2021}{}}%
\CSLLeftMargin{17. }%
\CSLRightInline{Khubchandani J, Sharma S, Price JH, Wiblishauser MJ,
Sharma M, Webb FJ. {COVID}-19 vaccination hesitancy in the united
states: A rapid national assessment. \emph{Journal of Community Health}.
2021;46(2):270-277.
doi:\href{https://doi.org/10.1007/s10900-020-00958-x}{10.1007/s10900-020-00958-x}}

\leavevmode\vadjust pre{\hypertarget{ref-guay2022}{}}%
\CSLLeftMargin{18. }%
\CSLRightInline{Guay M, Maquiling A, Chen R, et al. Measuring
inequalities in {COVID}-19 vaccination uptake and intent: Results from
the canadian community health survey 2021. \emph{{BMC} Public Health}.
2022;22(1).
doi:\href{https://doi.org/10.1186/s12889-022-14090-z}{10.1186/s12889-022-14090-z}}

\leavevmode\vadjust pre{\hypertarget{ref-muhajarine2021}{}}%
\CSLLeftMargin{19. }%
\CSLRightInline{Muhajarine N, Adeyinka DA, McCutcheon J, Green KL,
Fahlman M, Kallio N. {COVID}-19 vaccine hesitancy and refusal and
associated factors in an adult population in saskatchewan, canada:
Evidence from predictive modelling. Gesser-Edelsburg A, ed. \emph{{PLOS}
{ONE}}. 2021;16(11):e0259513.
doi:\href{https://doi.org/10.1371/journal.pone.0259513}{10.1371/journal.pone.0259513}}

\leavevmode\vadjust pre{\hypertarget{ref-hussain2022}{}}%
\CSLLeftMargin{20. }%
\CSLRightInline{Hussain B, Latif A, Timmons S, Nkhoma K, Nellums LB.
Overcoming {COVID}-19 vaccine hesitancy among ethnic minorities: A
systematic review of {UK} studies. \emph{Vaccine}.
2022;40(25):3413-3432.
doi:\href{https://doi.org/10.1016/j.vaccine.2022.04.030}{10.1016/j.vaccine.2022.04.030}}

\leavevmode\vadjust pre{\hypertarget{ref-mosby2021}{}}%
\CSLLeftMargin{21. }%
\CSLRightInline{Mosby I, Swidrovich J. Medical experimentation and the
roots of {COVID}-19 vaccine hesitancy among indigenous peoples in
canada. \emph{Canadian Medical Association Journal}.
2021;193(11):E381-E383.
doi:\href{https://doi.org/10.1503/cmaj.210112}{10.1503/cmaj.210112}}

\leavevmode\vadjust pre{\hypertarget{ref-bogart2021}{}}%
\CSLLeftMargin{22. }%
\CSLRightInline{Bogart LM, Ojikutu BO, Tyagi K, et al. {COVID}-19
related medical mistrust, health impacts, and potential vaccine
hesitancy among black americans living with {HIV}. \emph{{JAIDS} Journal
of Acquired Immune Deficiency Syndromes}. 2021;86(2):200-207.
doi:\href{https://doi.org/10.1097/qai.0000000000002570}{10.1097/qai.0000000000002570}}

\leavevmode\vadjust pre{\hypertarget{ref-freeman2020}{}}%
\CSLLeftMargin{23. }%
\CSLRightInline{Freeman D, Loe BS, Chadwick A, et al. {COVID}-19 vaccine
hesitancy in the {UK}: The oxford coronavirus explanations, attitudes,
and narratives survey (oceans) {II}. \emph{Psychological Medicine}.
2020;52(14):3127-3141.
doi:\href{https://doi.org/10.1017/s0033291720005188}{10.1017/s0033291720005188}}

\leavevmode\vadjust pre{\hypertarget{ref-malik2020}{}}%
\CSLLeftMargin{24. }%
\CSLRightInline{Malik AA, McFadden SM, Elharake J, Omer SB. Determinants
of {COVID}-19 vaccine acceptance in the {US}.
\emph{{EClinicalMedicine}}. 2020;26:100495.
doi:\href{https://doi.org/10.1016/j.eclinm.2020.100495}{10.1016/j.eclinm.2020.100495}}

\leavevmode\vadjust pre{\hypertarget{ref-nguyen2021}{}}%
\CSLLeftMargin{25. }%
\CSLRightInline{Nguyen KH, Nguyen K, Corlin L, Allen JD, Chung M.
Changes in {COVID}-19 vaccination receipt and intention to vaccinate by
socioeconomic characteristics and geographic area, united states,
january 6 {\textendash} march 29, 2021. \emph{Annals of Medicine}.
2021;53(1):1419-1428.
doi:\href{https://doi.org/10.1080/07853890.2021.1957998}{10.1080/07853890.2021.1957998}}

\leavevmode\vadjust pre{\hypertarget{ref-mollalo2021}{}}%
\CSLLeftMargin{26. }%
\CSLRightInline{Mollalo A, Tatar M. Spatial modeling of {COVID}-19
vaccine hesitancy in the united states. \emph{International Journal of
Environmental Research and Public Health}. 2021;18(18):9488.
doi:\href{https://doi.org/10.3390/ijerph18189488}{10.3390/ijerph18189488}}

\leavevmode\vadjust pre{\hypertarget{ref-yang2022}{}}%
\CSLLeftMargin{27. }%
\CSLRightInline{Yang TC, Matthews SA, Sun F. Multiscale dimensions of
spatial process: {COVID}-19 fully vaccinated rates in u.s. counties.
\emph{American Journal of Preventive Medicine}. 2022;63(6):954-961.
doi:\href{https://doi.org/10.1016/j.amepre.2022.06.006}{10.1016/j.amepre.2022.06.006}}

\leavevmode\vadjust pre{\hypertarget{ref-tiu2022}{}}%
\CSLLeftMargin{28. }%
\CSLRightInline{Tiu A, Susswein Z, Merritt A, Bansal S. Characterizing
the spatiotemporal heterogeneity of the {COVID}-19 vaccination
landscape. \emph{American Journal of Epidemiology}.
2022;191(10):1792-1802.
doi:\href{https://doi.org/10.1093/aje/kwac080}{10.1093/aje/kwac080}}

\leavevmode\vadjust pre{\hypertarget{ref-bhuiyan2022}{}}%
\CSLLeftMargin{29. }%
\CSLRightInline{Bhuiyan MAN, Davis TC, Arnold CL, et al. Using the
social vulnerability index to assess {COVID}-19 vaccine uptake in
louisiana. \emph{{GeoJournal}}. Published online December 2022.
doi:\href{https://doi.org/10.1007/s10708-022-10802-5}{10.1007/s10708-022-10802-5}}

\leavevmode\vadjust pre{\hypertarget{ref-wood2022}{}}%
\CSLLeftMargin{30. }%
\CSLRightInline{Wood AJ, MacKintosh AM, Stead M, Kao RR. Predicting
future spatial patterns in {COVID}-19 booster vaccine uptake. Published
online September 2022.
doi:\href{https://doi.org/10.1101/2022.08.30.22279415}{10.1101/2022.08.30.22279415}}

\leavevmode\vadjust pre{\hypertarget{ref-choi2021}{}}%
\CSLLeftMargin{31. }%
\CSLRightInline{Choi KH, Denice PA, Ramaj S. Vaccine and {COVID}-19
trajectories. \emph{Socius: Sociological Research for a Dynamic World}.
2021;7:237802312110529.
doi:\href{https://doi.org/10.1177/23780231211052946}{10.1177/23780231211052946}}

\leavevmode\vadjust pre{\hypertarget{ref-mckinnon2021}{}}%
\CSLLeftMargin{32. }%
\CSLRightInline{McKinnon B, Quach C, Dubé Ève, Nguyen CT, Zinszer K.
Social inequalities in {COVID}-19 vaccine acceptance and uptake for
children and adolescents in montreal, canada. \emph{Vaccine}.
2021;39(49):7140-7145.
doi:\href{https://doi.org/10.1016/j.vaccine.2021.10.077}{10.1016/j.vaccine.2021.10.077}}

\leavevmode\vadjust pre{\hypertarget{ref-tsasis2012}{}}%
\CSLLeftMargin{33. }%
\CSLRightInline{Tsasis P, Evans JM, Owen S. Reframing the challenges to
integrated care: A complex-adaptive systems perspective.
\emph{International Journal of Integrated Care}. 2012;12(5).
doi:\href{https://doi.org/10.5334/ijic.843}{10.5334/ijic.843}}

\leavevmode\vadjust pre{\hypertarget{ref-muratov2018}{}}%
\CSLLeftMargin{34. }%
\CSLRightInline{Muratov S, Lee J, Holbrook A, et al. Regional variation
in healthcare spending and mortality among senior high-cost healthcare
users in ontario, canada: A retrospective matched cohort study.
\emph{{BMC} Geriatrics}. 2018;18(1).
doi:\href{https://doi.org/10.1186/s12877-018-0952-7}{10.1186/s12877-018-0952-7}}

\leavevmode\vadjust pre{\hypertarget{ref-dong2022}{}}%
\CSLLeftMargin{35. }%
\CSLRightInline{Dong L, Sahu R, Black R. Governance in the
transformational journey toward integrated healthcare: The case of
ontario. \emph{Journal of Information Technology Teaching Cases}.
Published online December 2022:204388692211473.
doi:\href{https://doi.org/10.1177/20438869221147313}{10.1177/20438869221147313}}

\leavevmode\vadjust pre{\hypertarget{ref-lysyk2015}{}}%
\CSLLeftMargin{36. }%
\CSLRightInline{Auditor General of Ontario O of the, ed. Annual {R}eport
2015. In: \emph{Section 3.08: LHINs - Local Health Integration
Networks}. Queen's Printer for Ontario; 2015. Accessed May 12, 2023.
\url{https://www.auditor.on.ca/en/content/annualreports/arreports/en15/3.08en15.pdf}}

\leavevmode\vadjust pre{\hypertarget{ref-lysyk2016}{}}%
\CSLLeftMargin{37. }%
\CSLRightInline{Auditor General of Ontario O of the, ed. Annual {R}eport
2016. In: \emph{Section 3.03: Electronic Health Records' Implementation
Status}. Queen's Printer for Ontario; 2016. Accessed May 12, 2023.
\url{https://www.auditor.on.ca/en/content/annualreports/arreports/en16/v1_303en16.pdf}}

\leavevmode\vadjust pre{\hypertarget{ref-sethuram2023}{}}%
\CSLLeftMargin{38. }%
\CSLRightInline{Sethuram C, McCutcheon T, Liddy C. An environmental scan
of ontario health teams: A descriptive study. \emph{{BMC} Health
Services Research}. 2023;23(1).
doi:\href{https://doi.org/10.1186/s12913-023-09102-6}{10.1186/s12913-023-09102-6}}

\leavevmode\vadjust pre{\hypertarget{ref-nguyen2022}{}}%
\CSLLeftMargin{39. }%
\CSLRightInline{Nguyen KH, Anneser E, Toppo A, Allen JD, Parott JS,
Corlin L. Disparities in national and state estimates of {COVID}-19
vaccination receipt and intent to vaccinate by race/ethnicity, income,
and age group among adults~\(\geq\)~18~years, united states.
\emph{Vaccine}. 2022;40(1):107-113.
doi:\href{https://doi.org/10.1016/j.vaccine.2021.11.040}{10.1016/j.vaccine.2021.11.040}}

\leavevmode\vadjust pre{\hypertarget{ref-shih2021}{}}%
\CSLLeftMargin{40. }%
\CSLRightInline{Shih SF, Wagner AL, Masters NB, Prosser LA, Lu Y,
Zikmund-Fisher BJ. Vaccine hesitancy and rejection of a vaccine for the
novel coronavirus in the united states. \emph{Frontiers in Immunology}.
2021;12.
doi:\href{https://doi.org/10.3389/fimmu.2021.558270}{10.3389/fimmu.2021.558270}}

\leavevmode\vadjust pre{\hypertarget{ref-cnat2022a}{}}%
\CSLLeftMargin{41. }%
\CSLRightInline{Cénat JM, Noorishad PG, Farahi SMMM, et al. Prevalence
and factors related to {COVID}-19 vaccine hesitancy and unwillingness in
canada: A systematic review and meta-analysis. \emph{Journal of Medical
Virology}. 2022;95(1).
doi:\href{https://doi.org/10.1002/jmv.28156}{10.1002/jmv.28156}}

\leavevmode\vadjust pre{\hypertarget{ref-deming1940}{}}%
\CSLLeftMargin{42. }%
\CSLRightInline{Deming WE, Stephan FF. On a least squares adjustment of
a sampled frequency table when the expected marginal totals are known.
\emph{The Annals of Mathematical Statistics}. 1940;11(4):427-444.
doi:\href{https://doi.org/10.1214/aoms/1177731829}{10.1214/aoms/1177731829}}

\leavevmode\vadjust pre{\hypertarget{ref-lumley2011}{}}%
\CSLLeftMargin{43. }%
\CSLRightInline{Lumley T. \emph{Complex Surveys}. John Wiley \& Sons;
2011.}

\leavevmode\vadjust pre{\hypertarget{ref-wickham2019}{}}%
\CSLLeftMargin{44. }%
\CSLRightInline{Wickham H, Averick M, Bryan J, et al. Welcome to the
{tidyverse}. \emph{Journal of Open Source Software}. 2019;4(43):1686.
doi:\href{https://doi.org/10.21105/joss.01686}{10.21105/joss.01686}}

\leavevmode\vadjust pre{\hypertarget{ref-quarto}{}}%
\CSLLeftMargin{45. }%
\CSLRightInline{Allaire J. \emph{Quarto: R Interface to 'Quarto'
Markdown Publishing System}.; 2022.
\url{https://CRAN.R-project.org/package=quarto}}

\leavevmode\vadjust pre{\hypertarget{ref-modelsummary}{}}%
\CSLLeftMargin{46. }%
\CSLRightInline{Arel-Bundock V. {modelsummary}: Data and model summaries
in {R}. \emph{Journal of Statistical Software}. 2022;103(1):1-23.
doi:\href{https://doi.org/10.18637/jss.v103.i01}{10.18637/jss.v103.i01}}

\leavevmode\vadjust pre{\hypertarget{ref-gtsummary}{}}%
\CSLLeftMargin{47. }%
\CSLRightInline{Sjoberg DD, Whiting K, Curry M, Lavery JA, Larmarange J.
Reproducible summary tables with the gtsummary package. \emph{{The R
Journal}}. 2021;13:570-580.
doi:\href{https://doi.org/10.32614/RJ-2021-053}{10.32614/RJ-2021-053}}

\leavevmode\vadjust pre{\hypertarget{ref-carter2022}{}}%
\CSLLeftMargin{48. }%
\CSLRightInline{Carter MA, Biro S, Maier A, Shingler C, Guan TH.
{COVID}-19 vaccine uptake in southeastern ontario, canada: Monitoring
and addressing health inequities. \emph{Journal of Public Health
Management and Practice}. 2022;28(6):615-623.
doi:\href{https://doi.org/10.1097/phh.0000000000001565}{10.1097/phh.0000000000001565}}

\leavevmode\vadjust pre{\hypertarget{ref-basta2022}{}}%
\CSLLeftMargin{49. }%
\CSLRightInline{Basta NE, Sohel N, Sulis G, et al. Factors associated
with willingness to receive a {COVID}-19 vaccine among 23, 819 adults
aged 50 years or older: An analysis of the canadian longitudinal study
on aging. \emph{American Journal of Epidemiology}. 2022;191(6):987-998.
doi:\href{https://doi.org/10.1093/aje/kwac029}{10.1093/aje/kwac029}}

\leavevmode\vadjust pre{\hypertarget{ref-cnat2022b}{}}%
\CSLLeftMargin{50. }%
\CSLRightInline{Cénat JM, Noorishad PG, Bakombo SM, et al. A systematic
review on vaccine hesitancy in black communities in canada: Critical
issues and research failures. \emph{Vaccines}. 2022;10(11):1937.
doi:\href{https://doi.org/10.3390/vaccines10111937}{10.3390/vaccines10111937}}

\leavevmode\vadjust pre{\hypertarget{ref-cnat2023}{}}%
\CSLLeftMargin{51. }%
\CSLRightInline{Cénat JM, Farahi SMMM, Bakombo SM, et al. Vaccine
mistrust among black individuals in canada: The major role of health
literacy, conspiracy theories, and racial discrimination in the
healthcare system. \emph{Journal of Medical Virology}. 2023;95(4).
doi:\href{https://doi.org/10.1002/jmv.28738}{10.1002/jmv.28738}}

\leavevmode\vadjust pre{\hypertarget{ref-njoku2021}{}}%
\CSLLeftMargin{52. }%
\CSLRightInline{Njoku A, Joseph M, Felix R. Changing the narrative:
Structural barriers and racial and ethnic inequities in {COVID}-19
vaccination. \emph{International Journal of Environmental Research and
Public Health}. 2021;18(18):9904.
doi:\href{https://doi.org/10.3390/ijerph18189904}{10.3390/ijerph18189904}}

\leavevmode\vadjust pre{\hypertarget{ref-iveniuk2021}{}}%
\CSLLeftMargin{53. }%
\CSLRightInline{Iveniuk J, Leon S. \emph{Uneven Recovery: Measuring
Covid-19 Vaccine Equity in Ontario}. Wellesley Institute; 2021. Accessed
May 12, 2023.
\url{https://www.wellesleyinstitute.com/wp-content/uploads/2021/04/An-uneven-recovery-Measuring-COVID-19-vaccine-equity-in-Ontario.pdf}}

\leavevmode\vadjust pre{\hypertarget{ref-gill2022}{}}%
\CSLLeftMargin{54. }%
\CSLRightInline{Gill M, Datta D, Gregory P, Austin Z. {COVID}-19
vaccination in high-risk communities: Case study of brampton, ontario.
\emph{Canadian Pharmacists Journal / Revue des Pharmaciens du Canada}.
2022;155(6):345-351.
doi:\href{https://doi.org/10.1177/17151635221123042}{10.1177/17151635221123042}}

\leavevmode\vadjust pre{\hypertarget{ref-hawkins2020}{}}%
\CSLLeftMargin{55. }%
\CSLRightInline{Hawkins D. Differential occupational risk for {COVID}-19
and other infection exposure according to race and ethnicity.
\emph{American Journal of Industrial Medicine}. 2020;63(9):817-820.
doi:\href{https://doi.org/10.1002/ajim.23145}{10.1002/ajim.23145}}

\leavevmode\vadjust pre{\hypertarget{ref-ct2021}{}}%
\CSLLeftMargin{56. }%
\CSLRightInline{Côté D, Durant S, MacEachen E, et al. A rapid scoping
review of {COVID}-19 and vulnerable workers: Intersecting occupational
and public health issues. \emph{American Journal of Industrial
Medicine}. 2021;64(7):551-566.
doi:\href{https://doi.org/10.1002/ajim.23256}{10.1002/ajim.23256}}

\leavevmode\vadjust pre{\hypertarget{ref-mishra2021}{}}%
\CSLLeftMargin{57. }%
\CSLRightInline{Mishra S, Stall NM, Ma H, et al. \emph{A Vaccination
Strategy for Ontario {COVID}-19 Hotspots and Essential Workers}. Ontario
{COVID}-19 Science Advisory Table; 2021.
doi:\href{https://doi.org/10.47326/ocsat.2021.02.26.1.0}{10.47326/ocsat.2021.02.26.1.0}}

\leavevmode\vadjust pre{\hypertarget{ref-nguyen2021b}{}}%
\CSLLeftMargin{58. }%
\CSLRightInline{Nguyen KH, Yankey D, Coy KC, et al. {COVID}-19
vaccination coverage, intent, knowledge, attitudes, and beliefs among
essential workers, united states. \emph{Emerging Infectious Diseases}.
2021;27(11):2908-2913.
doi:\href{https://doi.org/10.3201/eid2711.211557}{10.3201/eid2711.211557}}

\leavevmode\vadjust pre{\hypertarget{ref-shah2019}{}}%
\CSLLeftMargin{59. }%
\CSLRightInline{Shah TI, Clark AF, Seabrook JA, Sibbald S, Gilliland JA.
Geographic accessibility to primary care providers: Comparing rural and
urban areas in southwestern ontario. \emph{The Canadian Geographer / Le
G{é}ographe canadien}. 2019;64(1):65-78.
doi:\href{https://doi.org/10.1111/cag.12557}{10.1111/cag.12557}}

\leavevmode\vadjust pre{\hypertarget{ref-crighton2015}{}}%
\CSLLeftMargin{60. }%
\CSLRightInline{Crighton EJ, Ragetlie R, Luo J, To T, Gershon A. A
spatial analysis of {COPD} prevalence, incidence, mortality and health
service use in ontario. \emph{Health Rep}. 2015;26(3):10-18.}

\leavevmode\vadjust pre{\hypertarget{ref-timony2022}{}}%
\CSLLeftMargin{61. }%
\CSLRightInline{Timony P, Houle SKD, Gauthier A, Waite NM. Geographic
distribution of ontario pharmacists: A focus on rural and northern
communities. \emph{Canadian Pharmacists Journal / Revue des Pharmaciens
du Canada}. 2022;155(5):267-276.
doi:\href{https://doi.org/10.1177/17151635221115411}{10.1177/17151635221115411}}

\leavevmode\vadjust pre{\hypertarget{ref-ontariohealth}{}}%
\CSLLeftMargin{62. }%
\CSLRightInline{\emph{{A}nnual {B}usiness {P}lan 2022/23}. Ontario
Health;
\url{https://www.ontariohealth.ca/sites/ontariohealth/files/2022-05/OHBusinessPlan22_23.pdf};
2022.}

\leavevmode\vadjust pre{\hypertarget{ref-smylie2022}{}}%
\CSLLeftMargin{63. }%
\CSLRightInline{Smylie J, McConkey S, Rachlis B, et al. Uncovering
{SARS}-{COV}-2 vaccine uptake and {COVID}-19 impacts among first
nations, inuit and m{é}tis peoples living in toronto and london,
ontario. \emph{Canadian Medical Association Journal}.
2022;194(29):E1018-E1026.
doi:\href{https://doi.org/10.1503/cmaj.212147}{10.1503/cmaj.212147}}

\leavevmode\vadjust pre{\hypertarget{ref-eissa2021}{}}%
\CSLLeftMargin{64. }%
\CSLRightInline{Eissa A, Lofters A, Akor N, Prescod C, Nnorom O.
Increasing {SARS}-{CoV}-2 vaccination rates among black people in
canada. \emph{Canadian Medical Association Journal}.
2021;193(31):E1220-E1221.
doi:\href{https://doi.org/10.1503/cmaj.210949}{10.1503/cmaj.210949}}

\leavevmode\vadjust pre{\hypertarget{ref-schafferderoo2020}{}}%
\CSLLeftMargin{65. }%
\CSLRightInline{DeRoo SS, Pudalov NJ, Fu LY. Planning for a {COVID}-19
vaccination program. \emph{{JAMA}}. 2020;323(24):2458.
doi:\href{https://doi.org/10.1001/jama.2020.8711}{10.1001/jama.2020.8711}}

\leavevmode\vadjust pre{\hypertarget{ref-stephenson2022}{}}%
\CSLLeftMargin{66. }%
\CSLRightInline{Stephenson M, Olson SM, Self WH, et al. Ascertainment of
vaccination status by self-report versus source documentation: Impact on
measuring {COVID}-19 vaccine effectiveness. \emph{Influenza and Other
Respiratory Viruses}. 2022;16(6):1101-1111.
doi:\href{https://doi.org/10.1111/irv.13023}{10.1111/irv.13023}}

\leavevmode\vadjust pre{\hypertarget{ref-ontario-covid}{}}%
\CSLLeftMargin{67. }%
\CSLRightInline{{Ontario COVID-19 Data Tool}. Accessed February 27,
2023.
\url{https://www.publichealthontario.ca/en/data-and-analysis/infectious-disease/covid-19-data-surveillance/covid-19-data-tool?tab=vaccine}}

\end{CSLReferences}



\end{document}
