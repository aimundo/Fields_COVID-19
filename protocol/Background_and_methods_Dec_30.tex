% Options for packages loaded elsewhere
\PassOptionsToPackage{unicode}{hyperref}
\PassOptionsToPackage{hyphens}{url}
\PassOptionsToPackage{dvipsnames,svgnames,x11names}{xcolor}
%
\documentclass[
  letterpaper,
  DIV=11,
  numbers=noendperiod]{scrartcl}

\usepackage{amsmath,amssymb}
\usepackage{lmodern}
\usepackage{iftex}
\ifPDFTeX
  \usepackage[T1]{fontenc}
  \usepackage[utf8]{inputenc}
  \usepackage{textcomp} % provide euro and other symbols
\else % if luatex or xetex
  \usepackage{unicode-math}
  \defaultfontfeatures{Scale=MatchLowercase}
  \defaultfontfeatures[\rmfamily]{Ligatures=TeX,Scale=1}
\fi
% Use upquote if available, for straight quotes in verbatim environments
\IfFileExists{upquote.sty}{\usepackage{upquote}}{}
\IfFileExists{microtype.sty}{% use microtype if available
  \usepackage[]{microtype}
  \UseMicrotypeSet[protrusion]{basicmath} % disable protrusion for tt fonts
}{}
\makeatletter
\@ifundefined{KOMAClassName}{% if non-KOMA class
  \IfFileExists{parskip.sty}{%
    \usepackage{parskip}
  }{% else
    \setlength{\parindent}{0pt}
    \setlength{\parskip}{6pt plus 2pt minus 1pt}}
}{% if KOMA class
  \KOMAoptions{parskip=half}}
\makeatother
\usepackage{xcolor}
\setlength{\emergencystretch}{3em} % prevent overfull lines
\setcounter{secnumdepth}{5}
% Make \paragraph and \subparagraph free-standing
\ifx\paragraph\undefined\else
  \let\oldparagraph\paragraph
  \renewcommand{\paragraph}[1]{\oldparagraph{#1}\mbox{}}
\fi
\ifx\subparagraph\undefined\else
  \let\oldsubparagraph\subparagraph
  \renewcommand{\subparagraph}[1]{\oldsubparagraph{#1}\mbox{}}
\fi


\providecommand{\tightlist}{%
  \setlength{\itemsep}{0pt}\setlength{\parskip}{0pt}}\usepackage{longtable,booktabs,array}
\usepackage{calc} % for calculating minipage widths
% Correct order of tables after \paragraph or \subparagraph
\usepackage{etoolbox}
\makeatletter
\patchcmd\longtable{\par}{\if@noskipsec\mbox{}\fi\par}{}{}
\makeatother
% Allow footnotes in longtable head/foot
\IfFileExists{footnotehyper.sty}{\usepackage{footnotehyper}}{\usepackage{footnote}}
\makesavenoteenv{longtable}
\usepackage{graphicx}
\makeatletter
\def\maxwidth{\ifdim\Gin@nat@width>\linewidth\linewidth\else\Gin@nat@width\fi}
\def\maxheight{\ifdim\Gin@nat@height>\textheight\textheight\else\Gin@nat@height\fi}
\makeatother
% Scale images if necessary, so that they will not overflow the page
% margins by default, and it is still possible to overwrite the defaults
% using explicit options in \includegraphics[width, height, ...]{}
\setkeys{Gin}{width=\maxwidth,height=\maxheight,keepaspectratio}
% Set default figure placement to htbp
\makeatletter
\def\fps@figure{htbp}
\makeatother
\newlength{\cslhangindent}
\setlength{\cslhangindent}{1.5em}
\newlength{\csllabelwidth}
\setlength{\csllabelwidth}{3em}
\newlength{\cslentryspacingunit} % times entry-spacing
\setlength{\cslentryspacingunit}{\parskip}
\newenvironment{CSLReferences}[2] % #1 hanging-ident, #2 entry spacing
 {% don't indent paragraphs
  \setlength{\parindent}{0pt}
  % turn on hanging indent if param 1 is 1
  \ifodd #1
  \let\oldpar\par
  \def\par{\hangindent=\cslhangindent\oldpar}
  \fi
  % set entry spacing
  \setlength{\parskip}{#2\cslentryspacingunit}
 }%
 {}
\usepackage{calc}
\newcommand{\CSLBlock}[1]{#1\hfill\break}
\newcommand{\CSLLeftMargin}[1]{\parbox[t]{\csllabelwidth}{#1}}
\newcommand{\CSLRightInline}[1]{\parbox[t]{\linewidth - \csllabelwidth}{#1}\break}
\newcommand{\CSLIndent}[1]{\hspace{\cslhangindent}#1}

\KOMAoption{captions}{tableheading}
\makeatletter
\makeatother
\makeatletter
\makeatother
\makeatletter
\@ifpackageloaded{caption}{}{\usepackage{caption}}
\AtBeginDocument{%
\ifdefined\contentsname
  \renewcommand*\contentsname{Table of contents}
\else
  \newcommand\contentsname{Table of contents}
\fi
\ifdefined\listfigurename
  \renewcommand*\listfigurename{List of Figures}
\else
  \newcommand\listfigurename{List of Figures}
\fi
\ifdefined\listtablename
  \renewcommand*\listtablename{List of Tables}
\else
  \newcommand\listtablename{List of Tables}
\fi
\ifdefined\figurename
  \renewcommand*\figurename{Figure}
\else
  \newcommand\figurename{Figure}
\fi
\ifdefined\tablename
  \renewcommand*\tablename{Table}
\else
  \newcommand\tablename{Table}
\fi
}
\@ifpackageloaded{float}{}{\usepackage{float}}
\floatstyle{ruled}
\@ifundefined{c@chapter}{\newfloat{codelisting}{h}{lop}}{\newfloat{codelisting}{h}{lop}[chapter]}
\floatname{codelisting}{Listing}
\newcommand*\listoflistings{\listof{codelisting}{List of Listings}}
\makeatother
\makeatletter
\@ifpackageloaded{caption}{}{\usepackage{caption}}
\@ifpackageloaded{subcaption}{}{\usepackage{subcaption}}
\makeatother
\makeatletter
\@ifpackageloaded{tcolorbox}{}{\usepackage[many]{tcolorbox}}
\makeatother
\makeatletter
\@ifundefined{shadecolor}{\definecolor{shadecolor}{rgb}{.97, .97, .97}}
\makeatother
\makeatletter
\makeatother
\ifLuaTeX
  \usepackage{selnolig}  % disable illegal ligatures
\fi
\IfFileExists{bookmark.sty}{\usepackage{bookmark}}{\usepackage{hyperref}}
\IfFileExists{xurl.sty}{\usepackage{xurl}}{} % add URL line breaks if available
\urlstyle{same} % disable monospaced font for URLs
\hypersetup{
  pdftitle={Covid Fields Data Protocol},
  pdfauthor={Ariel Mundo Ortiz},
  colorlinks=true,
  linkcolor={blue},
  filecolor={Maroon},
  citecolor={Blue},
  urlcolor={Blue},
  pdfcreator={LaTeX via pandoc}}

\title{Covid Fields Data Protocol}
\author{Ariel Mundo Ortiz}
\date{}

\begin{document}
\maketitle
\ifdefined\Shaded\renewenvironment{Shaded}{\begin{tcolorbox}[borderline west={3pt}{0pt}{shadecolor}, enhanced, interior hidden, sharp corners, frame hidden, boxrule=0pt, breakable]}{\end{tcolorbox}}\fi

\hypertarget{background}{%
\section{Background}\label{background}}

The COVID-19 pandemic continues around the world with more than 600
million confirmed cases as of November of
2022\textsuperscript{\protect\hyperlink{ref-WHO-Covid}{1}}. During the
first months of the pandemic in early 2020, non-pharmaceutical
interventions (e.g., masking, social distancing) were the only methods
available to manage the spread of the disease, but the rapid development
of vaccines against the virus permitted their approval and use in some
countries towards the last month part of 2020. For example, in the US
and Canada vaccine campaigns began in mid-December of
2020\textsuperscript{\protect\hyperlink{ref-tanne2020}{2},\protect\hyperlink{ref-bogoch2022}{3}}.
Although it has been estimated that vaccines against COVID-19 have
prevented around 14 millions of deaths
worldwide\textsuperscript{\protect\hyperlink{ref-watson2022}{4}}, the
rollout of COVID-19 vaccines has faced multiple challenges since its
inception.

In this regard, vaccination efforts have faced multiple: Inequalities
with regard to vaccine access due to socio-economic factors, vaccine
hesitancy, and differences in vaccination rates across different
segments of the population are among the challenges identified in the
administration of COVID-19
vaccines\textsuperscript{\protect\hyperlink{ref-gerretsen2021}{5}--\protect\hyperlink{ref-malik2020}{7}}.
In the case of Canada, lower vaccine uptake has been associated with
socio-economic factors such as younger age, educational level, presence
of children in the household, lack of a regular healthcare provider,
ethnic origin, and financial
instability\textsuperscript{\protect\hyperlink{ref-guay2022}{8}--\protect\hyperlink{ref-carter2022}{10}}.

Additionally, it has been shown that geography also plays a crucial role
in vaccination rates, as they vary due to spatial differences in
attitudes towards
vaccination\textsuperscript{\protect\hyperlink{ref-malik2020}{7}},
geographical differences in vaccine access and supply, vaccination
location availability, and lack of prioritization of vulnerable
groups\textsuperscript{\protect\hyperlink{ref-bogoch2022}{3},\protect\hyperlink{ref-nguyen2021}{11}}.

Studies that analyze geographical variations in vaccine uptake can help
inform public health decision-makers to design policies to that are
aimed at addressing vaccination disparities. In this regad, previous
geographical (spatial) analyses of vaccination rates have shown that
variations in vaccine uptake can occur within small governamental
administrative units (e.g., counties in the case of the
US)\textsuperscript{\protect\hyperlink{ref-mollalo2021}{12}--\protect\hyperlink{ref-bhuiyan2022}{15}},
and that geographical analyses can be predictive of booster uptake
patterns\textsuperscript{\protect\hyperlink{ref-wood2022}{16}}.

In Canada, studies that have used a spatial approach to analyze vaccine
uptake have shown disparities in vaccination rates across low and high
income neighborhoods in the city of
Toronto\textsuperscript{\protect\hyperlink{ref-choi2021}{17}}, among
adolescents from deprived neighborhoods in the city of
Montreal\textsuperscript{\protect\hyperlink{ref-mckinnon2021}{18}}, and
highlighted disparities in vaccination status depending on age, income,
and ethnic origin in all of the Canadian
provinces\textsuperscript{\protect\hyperlink{ref-guay2022}{8}}. However,
to the best of our knowledge, there are no studies that have analyzed
vaccination status within a province at the district/rural municipality
level, which can be useful to identify inequalities over these
geographical areas, thus providing a dissagregated view that can help
understand the barriers for vaccine delivery in the case of visible
minorities, which have been disproportionately impacted in the
pandemic\textsuperscript{\protect\hyperlink{ref-hussain2022}{19}}.

\hypertarget{research-question}{%
\section{Research Question}\label{research-question}}

This study will examine self-reported COVID-19 vaccination status in the
province of Ontario in order to determine the influence in vaccination
status due to socio-economic (e.g., ethnic origin, age, income) and
geographical factors (at the municipal level).

\hypertarget{methods}{%
\section{Methods}\label{methods}}

\hypertarget{data-source-survey-overview}{%
\subsection{Data source: survey
overview}\label{data-source-survey-overview}}

We obtained data from the Fields Institute for Research in Mathematical
Sciences' (henceforth Fields) \emph{Survey of COVID-19 related
Behaviours and Attitudes}, a repeated cross sectional survey focused on
the Canadian province of Ontario which ran from Sept 30, 2021 until
January 17, 2022. This survey was commissioned by Fields and the
Mathematical Modelling of COVID-19 Task Force, under the supervision of
Dr.~Kumar Murty, the Director of Fields with funding from the Canadian
Institutes of Health Research. The survey was conducted by a third-party
service provider (RIWI Corp.), under ethical guidance from University of
Toronto.

The survey was deployed using random domain intercept technology.
Briefly, when web users clicked on a registered but commercially
inactive web link or typed in a web address for a site that was dormant,
they had a random chance of that link being temporarily managed by the
company that administered the survey (RIWI Corp). Thus, instead of
coming across a notification about the status of the site(``this page
does not exist''), the survey was deployed to the user. Web users then
decided whether to anonymously participate, exiting the survey at any
time if
desired\textsuperscript{\protect\hyperlink{ref-sargent2022}{20}}.

Respondents who wished to participate were asked to select their age
from a matrix of values, and subsequent questions were displayed one at
a time, after the respondent confirmed their selection by answering and
selecting ``next''. Those who do not wished to participate were asked to
either close the browser window or navigate away from the domain. After
the survey closed (complete or incomplete) no one from that internet
protocol (IP) address could access the survey again and the domain entry
point rotated such that if a respondent were to attempt to access the
survey again, share the link, or enter via the same address using an
alternative IP address, the survey would not render.

Additionally, respondents who indicated they were under the age of 16
were exited from the survey. No record was created in this case and due
to domain cycling these users were unable to navigate back to the ``age
select'' screen. The personal identifier information from each
respondent was automatically scrubbed and replaced by a unique ID.
Respondents were drawn exclusively from the province of Ontario, as per
their devices meta-data.

\hypertarget{survey-responses}{%
\subsection{Survey responses}\label{survey-responses}}

\hypertarget{sec-socio-demographic-factors}{%
\subsubsection{Socio-demographic
factors}\label{sec-socio-demographic-factors}}

From the different answers provided by the survey respondents, we
selected the age group which they belonged to, income bracket,
race/ethnicity, and employment status. The original survey included
additional questions (e.g., sick leave, remote work, presence of minors
in the household) but the survey design, which permitted respondents to
exit the survey at any point resulted in a high rates of missing data
for most of these answers. The socio-economic factors chosen for this
study were the ones that had both the lowest rates of missingness and
that provided an adequate level socio-economic and demographic
information for our analysis. Information about the chosen
socio-economic factors from the survey is provided in
Table~\ref{tbl-covariates}.

\hypertarget{tbl-covariates}{}
\begin{longtable}[]{@{}
  >{\raggedright\arraybackslash}p{(\columnwidth - 2\tabcolsep) * \real{0.1457}}
  >{\raggedright\arraybackslash}p{(\columnwidth - 2\tabcolsep) * \real{0.8543}}@{}}
\caption{\label{tbl-covariates}Selected socio-economic factors from the
survey}\tabularnewline
\toprule()
\begin{minipage}[b]{\linewidth}\raggedright
Variable
\end{minipage} & \begin{minipage}[b]{\linewidth}\raggedright
Values
\end{minipage} \\
\midrule()
\endfirsthead
\toprule()
\begin{minipage}[b]{\linewidth}\raggedright
Variable
\end{minipage} & \begin{minipage}[b]{\linewidth}\raggedright
Values
\end{minipage} \\
\midrule()
\endhead
Age group & 15-24,25-34,35-44,45-54,55-64, 65+ \\
Income bracket (CAD) & \textless15,000, 15,000-24,999, 25,000-39,999,
40,000-59,999, 60,000-89,999, \textgreater90,000 \\
Race/ethnicity & Arab/Middle Eastern, Black, East Asian/Pacific
Islander, Indigenous, Latin American, Mixed, South Asian, White
Caucasian, other \\
Employment status & yes, no \\
\bottomrule()
\end{longtable}

\hypertarget{vaccination-status}{%
\subsubsection{Vaccination status}\label{vaccination-status}}

We selected two of the questions that were asked in the survey regarding
vaccination status:

\begin{itemize}
\item
  ``Have you received the first dose of the COVID vaccine?'', with
  possible answers ``yes'' and ``no''
\item
  (If answered ``yes'' to the previous question) ``Have you received the
  second dose of the COVID vaccine?'' with possible answers ``yes'' and
  ``no''
\end{itemize}

\hypertarget{data-cleaning}{%
\subsection{Data cleaning}\label{data-cleaning}}

The original dataset contained 39,029 entries (where each entry
corresponded to a set of answers provided by a unique respondent).
Following a preliminary analysis to identify the missing rates across
the different answers within each entry, it was identified that many of
the answers had high missing rates (\textgreater80\%) (note that the
graph will be in the Appendix). Therefore, the dataset was cleaned in
order to contain only the independent variables of interest with the
lowest missing rates (Table~\ref{tbl-covariates}) and the dependent
variables.

The cleaning process also included removing outliers that were
identified during the preliminary analyses. Specifically, we removed
those respondents that indicated to be below 25 years of age, living in
a household of size 1, and that reported an income above CAD 110,000.
After cleaning the dataset contained 5,247 entries (unique respondents).

\hypertarget{corrections}{%
\subsection{Corrections}\label{corrections}}

Differences between the clean dataset and the 2016 Census data for
Ontario regarding proportions of age groups, income, and ethnicity/race
were identified. Therefore, the proportions of each of these variables
were corrected using an iterative proportional fitting procedure (also
known as
\emph{raking})\textsuperscript{\protect\hyperlink{ref-deming1940}{21}}
in R using the \texttt{survey} package. Proportions for the correction
were obtained from the 2016 Census Data for Ontario. Because the
categories provided by the survey in some cases (e.g., race/ethnicity
categories) did not match the categories from the Census, we aggregated
some of them to obtain an approximation to the categories in the Census.
The aggregation is described in detail in the Appendix.

\hypertarget{sec-geographical-location}{%
\subsection{Geographical location}\label{sec-geographical-location}}

For each survey participant certain data was automatically captured.
This included the nearest municipality, which resulted in a total of 578
different municipalities within the dataset. Because our interest lies
in exploring vaccination status within a governamental administrative
unit that encompasses multiple cities, we grouped the municipalities
from the dataset according to the regional government aggregations
provided by the Association of Municipalities of Ontario, which groups
municipalities within regions, counties, districts, and
single-tiers\textsuperscript{\protect\hyperlink{ref-ontario-municipalities}{22}}.

\hypertarget{statistical-model}{%
\subsection{Statistical model}\label{statistical-model}}

We used a multivariable logistic regression with the socio-economic
factors described in Section~\ref{sec-socio-demographic-factors} and the
aggregated demographical locations from
Section~\ref{sec-geographical-location}.

\hypertarget{references}{%
\subsection{References}\label{references}}

\hypertarget{refs}{}
\begin{CSLReferences}{0}{0}
\leavevmode\vadjust pre{\hypertarget{ref-WHO-Covid}{}}%
\CSLLeftMargin{1. }%
\CSLRightInline{{World Health Organization Coronavirus (COVID-19)
Dashboard}. Accessed November 27, 2022. \url{https://covid19.who.int/}}

\leavevmode\vadjust pre{\hypertarget{ref-tanne2020}{}}%
\CSLLeftMargin{2. }%
\CSLRightInline{Tanne JH. Covid-19: {FDA} panel votes to authorise
pfizer {BioNTech} vaccine. \emph{{BMJ}}. Published online December
2020:m4799.
doi:\href{https://doi.org/10.1136/bmj.m4799}{10.1136/bmj.m4799}}

\leavevmode\vadjust pre{\hypertarget{ref-bogoch2022}{}}%
\CSLLeftMargin{3. }%
\CSLRightInline{Bogoch II, Halani S. {COVID}-19 vaccines: A geographic,
social and policy view of vaccination efforts in ontario, canada.
\emph{Cambridge Journal of Regions, Economy and Society}. Published
online November 2022.
doi:\href{https://doi.org/10.1093/cjres/rsac043}{10.1093/cjres/rsac043}}

\leavevmode\vadjust pre{\hypertarget{ref-watson2022}{}}%
\CSLLeftMargin{4. }%
\CSLRightInline{Watson OJ, Barnsley G, Toor J, Hogan AB, Winskill P,
Ghani AC. Global impact of the first year of {COVID}-19 vaccination: A
mathematical modelling study. \emph{The Lancet Infectious Diseases}.
2022;22(9):1293-1302.
doi:\href{https://doi.org/10.1016/s1473-3099(22)00320-6}{10.1016/s1473-3099(22)00320-6}}

\leavevmode\vadjust pre{\hypertarget{ref-gerretsen2021}{}}%
\CSLLeftMargin{5. }%
\CSLRightInline{Gerretsen P, Kim J, Caravaggio F, et al. Individual
determinants of {COVID}-19 vaccine hesitancy. Inbaraj LR, ed.
\emph{{PLOS} {ONE}}. 2021;16(11):e0258462.
doi:\href{https://doi.org/10.1371/journal.pone.0258462}{10.1371/journal.pone.0258462}}

\leavevmode\vadjust pre{\hypertarget{ref-nafilyan2021}{}}%
\CSLLeftMargin{6. }%
\CSLRightInline{Nafilyan V, Dolby T, Razieh C, et al. Sociodemographic
inequality in {COVID}-19 vaccination coverage among elderly adults in
england: A national linked data study. \emph{{BMJ} Open}.
2021;11(7):e053402.
doi:\href{https://doi.org/10.1136/bmjopen-2021-053402}{10.1136/bmjopen-2021-053402}}

\leavevmode\vadjust pre{\hypertarget{ref-malik2020}{}}%
\CSLLeftMargin{7. }%
\CSLRightInline{Malik AA, McFadden SM, Elharake J, Omer SB. Determinants
of {COVID}-19 vaccine acceptance in the {US}.
\emph{{EClinicalMedicine}}. 2020;26:100495.
doi:\href{https://doi.org/10.1016/j.eclinm.2020.100495}{10.1016/j.eclinm.2020.100495}}

\leavevmode\vadjust pre{\hypertarget{ref-guay2022}{}}%
\CSLLeftMargin{8. }%
\CSLRightInline{Guay M, Maquiling A, Chen R, et al. Measuring
inequalities in {COVID}-19 vaccination uptake and intent: Results from
the canadian community health survey 2021. \emph{{BMC} Public Health}.
2022;22(1).
doi:\href{https://doi.org/10.1186/s12889-022-14090-z}{10.1186/s12889-022-14090-z}}

\leavevmode\vadjust pre{\hypertarget{ref-muhajarine2021}{}}%
\CSLLeftMargin{9. }%
\CSLRightInline{Muhajarine N, Adeyinka DA, McCutcheon J, Green KL,
Fahlman M, Kallio N. {COVID}-19 vaccine hesitancy and refusal and
associated factors in an adult population in saskatchewan, canada:
Evidence from predictive modelling. Gesser-Edelsburg A, ed. \emph{{PLOS}
{ONE}}. 2021;16(11):e0259513.
doi:\href{https://doi.org/10.1371/journal.pone.0259513}{10.1371/journal.pone.0259513}}

\leavevmode\vadjust pre{\hypertarget{ref-carter2022}{}}%
\CSLLeftMargin{10. }%
\CSLRightInline{Carter MA, Biro S, Maier A, Shingler C, Guan TH.
{COVID}-19 vaccine uptake in southeastern ontario, canada: Monitoring
and addressing health inequities. \emph{Journal of Public Health
Management and Practice}. 2022;28(6):615-623.
doi:\href{https://doi.org/10.1097/phh.0000000000001565}{10.1097/phh.0000000000001565}}

\leavevmode\vadjust pre{\hypertarget{ref-nguyen2021}{}}%
\CSLLeftMargin{11. }%
\CSLRightInline{Nguyen KH, Nguyen K, Corlin L, Allen JD, Chung M.
Changes in {COVID}-19 vaccination receipt and intention to vaccinate by
socioeconomic characteristics and geographic area, united states,
january 6 {\textendash} march 29, 2021. \emph{Annals of Medicine}.
2021;53(1):1419-1428.
doi:\href{https://doi.org/10.1080/07853890.2021.1957998}{10.1080/07853890.2021.1957998}}

\leavevmode\vadjust pre{\hypertarget{ref-mollalo2021}{}}%
\CSLLeftMargin{12. }%
\CSLRightInline{Mollalo A, Tatar M. Spatial modeling of {COVID}-19
vaccine hesitancy in the united states. \emph{International Journal of
Environmental Research and Public Health}. 2021;18(18):9488.
doi:\href{https://doi.org/10.3390/ijerph18189488}{10.3390/ijerph18189488}}

\leavevmode\vadjust pre{\hypertarget{ref-yang2022}{}}%
\CSLLeftMargin{13. }%
\CSLRightInline{Yang TC, Matthews SA, Sun F. Multiscale dimensions of
spatial process: {COVID}-19 fully vaccinated rates in u.s. counties.
\emph{American Journal of Preventive Medicine}. 2022;63(6):954-961.
doi:\href{https://doi.org/10.1016/j.amepre.2022.06.006}{10.1016/j.amepre.2022.06.006}}

\leavevmode\vadjust pre{\hypertarget{ref-tiu2022}{}}%
\CSLLeftMargin{14. }%
\CSLRightInline{Tiu A, Susswein Z, Merritt A, Bansal S. Characterizing
the spatiotemporal heterogeneity of the {COVID}-19 vaccination
landscape. \emph{American Journal of Epidemiology}.
2022;191(10):1792-1802.
doi:\href{https://doi.org/10.1093/aje/kwac080}{10.1093/aje/kwac080}}

\leavevmode\vadjust pre{\hypertarget{ref-bhuiyan2022}{}}%
\CSLLeftMargin{15. }%
\CSLRightInline{Bhuiyan MAN, Davis TC, Arnold CL, et al. Using the
social vulnerability index to assess {COVID}-19 vaccine uptake in
louisiana. \emph{{GeoJournal}}. Published online December 2022.
doi:\href{https://doi.org/10.1007/s10708-022-10802-5}{10.1007/s10708-022-10802-5}}

\leavevmode\vadjust pre{\hypertarget{ref-wood2022}{}}%
\CSLLeftMargin{16. }%
\CSLRightInline{Wood AJ, MacKintosh AM, Stead M, Kao RR. Predicting
future spatial patterns in {COVID}-19 booster vaccine uptake. Published
online September 2022.
doi:\href{https://doi.org/10.1101/2022.08.30.22279415}{10.1101/2022.08.30.22279415}}

\leavevmode\vadjust pre{\hypertarget{ref-choi2021}{}}%
\CSLLeftMargin{17. }%
\CSLRightInline{Choi KH, Denice PA, Ramaj S. Vaccine and {COVID}-19
trajectories. \emph{Socius: Sociological Research for a Dynamic World}.
2021;7:237802312110529.
doi:\href{https://doi.org/10.1177/23780231211052946}{10.1177/23780231211052946}}

\leavevmode\vadjust pre{\hypertarget{ref-mckinnon2021}{}}%
\CSLLeftMargin{18. }%
\CSLRightInline{McKinnon B, Quach C, Dubé Ève, Nguyen CT, Zinszer K.
Social inequalities in {COVID}-19 vaccine acceptance and uptake for
children and adolescents in montreal, canada. \emph{Vaccine}.
2021;39(49):7140-7145.
doi:\href{https://doi.org/10.1016/j.vaccine.2021.10.077}{10.1016/j.vaccine.2021.10.077}}

\leavevmode\vadjust pre{\hypertarget{ref-hussain2022}{}}%
\CSLLeftMargin{19. }%
\CSLRightInline{Hussain B, Latif A, Timmons S, Nkhoma K, Nellums LB.
Overcoming {COVID}-19 vaccine hesitancy among ethnic minorities: A
systematic review of {UK} studies. \emph{Vaccine}.
2022;40(25):3413-3432.
doi:\href{https://doi.org/10.1016/j.vaccine.2022.04.030}{10.1016/j.vaccine.2022.04.030}}

\leavevmode\vadjust pre{\hypertarget{ref-sargent2022}{}}%
\CSLLeftMargin{20. }%
\CSLRightInline{Sargent RH, Laurie S, Weakland LF, et al. Use of random
domain intercept technology to track {COVID}-19 vaccination rates in
real time across the united states: Survey study. \emph{Journal of
Medical Internet Research}. 2022;24(7):e37920.
doi:\href{https://doi.org/10.2196/37920}{10.2196/37920}}

\leavevmode\vadjust pre{\hypertarget{ref-deming1940}{}}%
\CSLLeftMargin{21. }%
\CSLRightInline{Deming WE, Stephan FF. On a least squares adjustment of
a sampled frequency table when the expected marginal totals are known.
\emph{The Annals of Mathematical Statistics}. 1940;11(4):427-444.
doi:\href{https://doi.org/10.1214/aoms/1177731829}{10.1214/aoms/1177731829}}

\leavevmode\vadjust pre{\hypertarget{ref-ontario-municipalities}{}}%
\CSLLeftMargin{22. }%
\CSLRightInline{Ontario {M}unicipalities. Accessed December 30, 2022.
\url{https://www.amo.on.ca/about-us/municipal-101/ontario-municipalities}}

\end{CSLReferences}



\end{document}
